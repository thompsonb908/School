\section{Chapter 1}
\subsection{Long answer}
\begin{enumerate}
    \item Describe computer forensics in detail.\\
    Computer forensics is the use of analytical and investigative techniques to identify, collect, examine and preserve evidence/information which is magnetically stored or encoded.
    \item Who are the stake holders (users) of digital forensics?\\
    The military, government agencies, law firms, criminal prosecutors, academia, data recovery firms, corporations, insurance companies, individuals.
    \item What is digital evidence? Explain with examples.\\
    Information that has been processed and assembled so that it is relevant to an investigation and supports a specific finding or determination.
    Raw information is not in itself evidence.
    Real evidence is a physical object that someone can touch (ex: a laptop with fingerprints, USB).
    Documentary evidence is data stored as written matter, on paper or in electronic files (ex: email messages, logs, databases).
    Testimonial Evidence is information that forensic specialists use to support or interpret real or documentary evidence (ex: system access show that a user stored photographs on a desktop).
    Demonstrative evidence is information that helps explain other evidence (ex: chart explaining a technical concept to a judge).
    \item Describe types of digital forensics analysis.\\
    Disk forensics analyzes information on stored media such as computer hard drives.
    Email forensics study the source and content of email as evidence.
    Network forensics examines network traffic, including transaction logs and real-time monitoring.
    Internet forensics pieces where and when a use has been on the internet.
    Software forensics examines malicious computer code (also malware forensics).
    Live system forensics searches memory in real time, typically on compromised hosts to identify system abuse.
    Cell phone forensics searches the content of cell phones.
    \item Define anti-forensics and obscured information.\\
    
    \item Describe the Daubert Standard in detail.
\end{enumerate}

\subsection{Short Answer}
\begin{enumerate}
    \item Chain of custody
    \item Shut down machine
    \item In case other devices were connected
    \item What is the essence of the Daubert Standard?\\Only tools or techniques that have been accepted
    \item Preserve evidence integrity
    \item All of the above.
    \item 18 U.S.C\ () 1030, Fraud and related activity in connection with computeres
    \item The Pen register and 
    \item Data hiding
    \item Anti-forensic
    \item It cannot be changed
    \item Testimonial evidence
\end{enumerate}