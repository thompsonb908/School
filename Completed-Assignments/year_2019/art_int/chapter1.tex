\documentclass[a4paper]{article}

\usepackage[utf8]{inputenc}
\usepackage[T1]{fontenc}
\usepackage{textcomp}
\usepackage[english]{babel}
\usepackage{amsmath, amssymb}


%figure support
\usepackage{import}
\usepackage{xifthen}
\pdfminorversion=7
\usepackage{pdfpages}
\usepackage{transparent}
\newcommand{\incfig}[1]{%
	\def\svgwidth{\columnwidth}
	\import{./figures/}{#1.pdf_tex}
}

\pdfsuppresswarningpagegroup=1

\begin{document}
	\author{Brandon Thompson}
	\title{Chapter 1}
	\maketitle

	\medskip
	
	Artificial intelligence can be broken into two categories of thought.
	\begin{description}
		\item[Strong AI] 
			is the concept that given enough processing power and
			information a computer can think like a human.
			Strong AI has never been accomplished and is more of a long term sci-fi goal.
		
		\item[Weak AI]
                        is the alternative (and much more attainable) concept
                        to strong AI. Intelligent behavior can be modeled and follow by
                        a computer.
	\end{description}	
	
	To create AI, we use different types of methods to process the problems and the information.
	
	\begin{description}
		\item[Weak Methods]
			focus on inferencing through logic and automated reasoning. This does not
			necessarily include any real knowledge about the problem domain. Early
			research was entirely focused on this area.

		\item[Strong Methods] depend on the system being given a large amount of information
                        in the problem domain. Strong methods depend on weak methods because
                        we need a way to sort all of the information.
	\end{description}
	
	Production systems use a combination of weak method expert system shells to perform
	inference, but strong method rules to encode the knowledge.
	

\end{document}
