\documentclass[a4paper]{article}

\usepackage[utf8]{inputenc}
\usepackage[T1]{fontenc}
\usepackage{textcomp}
\usepackage[english]{babel}
\usepackage{amsmath, amssymb}
\usepackage{mathrsfs}

%figure support
\usepackage{import}
\usepackage{xifthen}
\pdfminorversion=7
\usepackage{pdfpages}
\usepackage{transparent}
\newcommand{\incfig}[1]{%
	\def\svgwidth{\columnwidth}
	\import{./figures/}{#1.pdf_tex}
}

\newtheorem{theorem}{Theorem}[section]
\newtheorem{corollary}{Corollary}[theorem]
\newtheorem{lemma}[theorem]{Lemma}

\pdfsuppresswarningpagegroup=1

\begin{document}
	\author{Brandon Thompson}
	\title{Block Ciphers}
	\maketitle
	
	\section{Stream Cipher vs Block Cipher}
	%key expansion from slide 4
	Stream ciphers encrypt one bit or byte at a time using the XOR operation.
	Block cipher is one where a section of plain text is treated as whole and used
	to produce a cipher text of equal length (typically 64 or 128 bits).\\

	\begin{description}
		\item[Feistal Cipher:] we can approximate the ideal block cipher by utilizing
			the concept of a product cipher, the execution of two or more simple
			ciphers in sequence such that the final result is is cryptographically
			stronger than any of the component ciphers. Goal is to develop a block 
			cipher with a key length of $k$ bits and block length of $n$ bits,
			allowing a total of $2^{k}$ possible transformations.
			\begin{itemize}
				\item \textbf{Substitution:} Each plain text element or group
					of elements is uniquely replaced by a corresponding
					cipher text element or group of elements.
				\item \textbf{Permutation:} A sequence of plain text elements is
					replaced by a permutation of that sequence. No
					elements are added, deleted, or replaced, in the sequence,
					rather the order of the elements is changed.
			\end{itemize}
	\end{description}
	Claude Shannon introduced the terms \textit{diffusion} and \textit{confusion} to capture
	the two basic building blocks for cryptography systems, to hinder statistical analysis
	of the cipher text. In \textbf{diffusion} the statistical structure of the plain text
	is dissipated into long-range statistics of the cipher text. This is achieved by having
	each plain text digit effect the value of many cipher text digits. An example is to
	encrypt a message $M = m_1,m_2,m_3,\ldots$ of characters with an averaging operation:
	\begin{center}
		\[y_n = \left( \sum_{i=1}^{k} m_{n+i} \right)\mod26\] 
	\end{center}
	adding $k$ successive letters to get a cipher text letter $y_n$. Letter frequencies
	in  the cipher text will be more nearly equal than in the plain text. In a binary
	block cipher, diffusion can be achieved by repeatedly performing some permutation on
	the data followed by applying a function to that permutation. \par
	Every block cipher involves a transformation of a block of plain text into a block
	of cipher text
	given functions $f_1,\ldots,f_d:\ \{0,1\}^{n} \to \{0,1\}^{n}$\\
	goal: build inevitable function $F:\{0,1\}^{2n} \to \{0,1\}^{2n}$ 
	%Feistal Network slide 12

	decryption
	

	\\\\
	\section{The Function}
	\begin{itemize}
		\item Pseudo Random Function {\bf{(PRF)}} defined over (K,X,Y):
	\end{itemize}
	\begin{center}
		F: K $\times$ X $\to $ Y
	\end{center}
	Such that exists ''efficient'' algorithm to evaluate F(k,x)\\
	%\hline
	\begin{itemize}
		\item Pseudo Random Permutation {\bf{(PRP)}} defined over (K,X):
	\end{itemize}
	\begin{center}
		E: K $\times$ X $\to$ Y
	\end{center}
	Such that:
	\begin{enumerate}
		\item Exists ''efficient'' \underline{deterministic} algorithm
			to evaluate E(k,x)
		\item The Function E(k,  $\cdot$) is one-to-one
		\item Exists ''efficient'' inversion algorithm D(k,y)
	\end{enumerate}
	Examples:
	\begin{itemize}
		\item 3DES: K $\times $ X $\to $ X where X $=\{0,1\}^{64}$, K $=\{0,1\}^{168}$ 
		\item AES: K $\times $ X $\to $ X where K $=$ X $= \{0,1\}^{128}$
	\end{itemize}
	Functionally, any PRP is also a PRF, A PRP is a PRF where X=Y and is efficiently invertible.\\
	\section{DES: The Data Encryption Standard}
	Given plain text block of $n$-bits and key of  $k$-bits, block ciphers
	encrypt/decrypt based on these block and key sizes.
	\begin{enumerate}
		\item 3DES: $n = 64$ bits, $k = 168$ bits
		\item AES: $n = 128$ bits, $k = 128, 192, 256$ bits
	\end{enumerate}
	Key is expanded to $k=k_1,\ldots,k_n$ message is encrypted using round
	functions $R(k,m)$, 3DES uses 48 iterations, AES-128 uses 10 iterations.
	\pagebreak
	\subsection{Feistel Network}
\begin{figure}[h]
	\centering 
   Given Functions $f_1, \ldots, f_d: \{0,1\}^{n}$\\
   Goal: build invertible function $F:\left\{ 0,1 \right\}^{2n} $
   \centering
   \incfig{feistal-network}
   \caption{Feistal Network}
   \label{fig:feistal-network}
\end{figure}

\begin{equation*}
	\text{Symbol Representation: }
	\begin{cases}
		R_i = f_i\left( R_{i-1} \right) \bigoplus L_{i-1}\\
		L_i = R_{i-1}
	\end{cases}
\end{equation*}

	For all $f_1, \ldots, f_d: \left\{ 0,1 \right\}^{n} \to  \left\{ 0,1 \right\}^{n} $
	Feistal network F: $\left\{ 0,1 \right\}^{2n} \to  \left\{ 0,1 \right\}^{2n} $ is invertible.
	Apply $f_1,\ldots,f_d$ in reverse order to decrypt. This process is used on
	many block ciphers, but not AES.\\\\
	Key is 56-bits, key expansion generates 16 keys of 48-bits.
	Uses S-box lookup table $S_i := \{0,1\}^{6} \to \{0,1\}^{4}$ 
	%slide 25
	If there is a linear relationship DES can be represented by matrix vector product.\\
	Thus the entire DES cipher would be linear.


	\section{Attack on Block Cipher}
	\subsection{Exhaustive Search}
	Goal: Given a few input pairs $\left( m_i, c_i = E\left( k_i, m_i \right)  \right) $
	$i = 1,\ldots,3$ find key $k$.
	\pagebreak
	\begin{lemma}
		Suppose DES is an \textbf{ideal cipher}\\
		$\left( 2^{56} \text{ random invertible functions} \right)$\\
		Then  $\forall\ m,c$ there is at most \textbf{one} key s.t.\ 
		$c=DES\left( k,m \right) $
	\end{lemma}
	\begin{center}
		\[
			Pr\left[ \exists\ k' \neq K, c = DES\left( k,m \right)
			= DES\left( k',m \right)  \right] \le
			\sum_{k'\in \left\{ 0,1 \right\}^{56} }^{}
			Pr\left[ DES\left( k,m \right) =\\ 
			DES\left( k',m \right)  \right] \le 2^{56} \cdot 
			\frac{1}{2^{64}} = \frac{1}{2^{8}}
		\]
	\end{center}
	For two DES pairs
	\[
		\left( m_1,c_1=DES\left( k,m_1 \right)  \right),
		\left( m_2,c_2=DES\left( k,m_2 \right)  \right) 
	\]
	Unicity probability $\approx 1-\frac{1}{2^{71}}$\\
	For AES-128: given two input/output pairs, unicity probability
	$\approx 1-\frac{1}{2^{128}}$ \\
	Two input/output pairs are enough for exhaustive key search.

	
	\subsection{DES Challenge}
	msg = "The unknown messages is: XXXX \ldots"\\
	CT  $=c_1,c_2,c_3,c_4$\\
	\textbf{Goal:} find  $k \in \{0,1\}^{56} \text{ s.t.\ } DES\left( k_i,m_i \right) =c_i \text{ for } i=1,2,3$\\\\
	56-bit ciphers should no be used, 128-bit key = $2^{72}$ days
	
	\section{Strengthen DES against exhaustive search}
	\subsection{Method 1: Triple DES}
	\begin{itemize}
		\item Let $E:K \times M \to M$ be a block cipher
		\item define 3E: $K^3\times M \to M$ as
			\begin{center}
				\[
				3E\left( \left( k_1,k_2,k_3 \right) ,m\right) 
				=E\left( k_1,D\left( k_2,E\left( k_3,m \right)  \right)  \right) \\
				k_1=k_2=k_3 \implies \text{single DES}
				\] 
			\end{center}
	\end{itemize}
	For 3DES: Key size = $3\times 56 = 168$ bits. $3\times $ slower than DES.
	Simple attack in time $=2^{118}$

	\subsection{Why not double DES?}
	\subsubsection{Meet in the middle attack}
	\begin{itemize}
		\item Define $2E\left( \left( k_1,k_2 \right) ,m \right) = E\left( k_1,E\left( k_2,m \right)  \right) $
		\item find $\left( k_1,k_2 \right) \text{ s.t.\ } E\left( k_1,E\left( k_2,m \right)  \right) =C$ \\
		\item equivocally: $E\left( k_2,m \right) = D\left( k_1,c \right) $ \\\\
		%slide 34 image
		two tables of keys and cipher texts can be combined to find both keys.\\\\
		space $\approx 2^{56}$
	\end{itemize}
	Attack:
	\begin{enumerate}
		\item Build table of $k_0,\ldots,k_n$ and $E\left( k_0,M \right),\ldots, E\left( k_n,M \right) $ 
		\item $\forall\ k \in \left\{ 0,1 \right\}^{56} $:
			test if $D\left( k,C \right) $ is in 2nd column.\\
			if so then $E\left( k_i,M \right) =D\left( k,C \right) \implies\ \left( k_i,k \right) = \left( k_2,k_1 \right) $
	\end{enumerate}
	time $=2^{56}\log\left( 2^{56} \right) + 2^{56}\log\left( 2^{56} \right) < 2^{63} \ll 2^{112}$ \\\\
	Same attack on 3DES: time $= 2^{118}$, space $= 2^{56}$.
	\subsection{Method 2: DESX}
	

\end{document}
