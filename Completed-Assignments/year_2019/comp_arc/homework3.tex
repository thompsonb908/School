\documentclass[a4paper]{article}

\usepackage[utf8]{inputenc}
\usepackage[T1]{fontenc}
\usepackage{textcomp}
\usepackage[english]{babel}
\usepackage{amsmath, amssymb}
\usepackage{multicol}

%figure support
\usepackage{import}
\usepackage{xifthen}
\pdfminorversion=7
\usepackage{pdfpages}
\usepackage{transparent}
\newcommand{\incfig}[1]{%
	\def\svgwidth{\columnwidth}
	\import{./figures/}{#1.pdf_tex}
}
\graphicspath{ {./figures/} }
\pdfsuppresswarningpagegroup=1

\begin{document}
	\title{EEL4768C.04 Homework 3 Due: 10/20/19}
	\author{Brandon Thompson 5517}
	\maketitle

	\begin{enumerate}
		\item Explain the organization of the classical Von Neuwmann machine and its
			major functional units.\\
			\\
			The Von Neumann architecture is a arithmetic logic unit (ALU), memory unit, 
			registers, control unit and inputs and outputs. Because the architecture is designed this 
			way, it allows for instructions and data to be stored in the same memory.
			\begin{description}
				\item[Arithmetic Logic Unit (ALU):] Allows for arithmetic operations and
					logic operations in the system.
				\item[Memory Unit:] RAM, primary/main memory. Directly accessible to the
					CPU. Loading from secondary memory into the RAM allows the CPU to
					operate much quicker.
				\item[Control Unit:] Controls operation of ALU, memory and input/output
					devices, telling them how to respond to instructions interpreted from
					the memory unit.
				\item[Registers:] High speed storage areas in the CPU. All data must be stored
					in a register before it can be processed.
			\end{description}
			%image for Von Neumann machine.
		\item Explain and compare the generation of control signals using hardwired
			or micro-programmed implementation for a control unit.\\
			\\
			The \textbf{hardwired control unit} uses logic gates to generate the output. This method is
			very fast for a limited number of operations. Because of how the hardwired control
			unit is implemented, making modifications is difficult and if one part is changed
			the system has to be rewired. Hardwired control units are known as Reduced Instruction
			Set Computers (RISC).\\
			The \textbf{Microprogrammed Control Unit} uses microinstructions in the control memory to
			to produce control signals. Because instructions require memory accesses the speed
			of operations is slower. Modifications are made by changing the microinstructions.
			Complex instructions can be broken into multiple smaller instructions. Used in
			Complex Instruction Set Computers (CISC).\\
			In general, hardwired control units are more difficult and costly to implement or
			alter than microprogrammed control units. Despite that, hardwired control units
			are more specialized and thus faster at what it was designed to do than microprogrammed
			control units.
	\end{enumerate}
\end{document}
