\documentclass[a4paper]{article}

\usepackage[utf8]{inputenc}
\usepackage[T1]{fontenc}
\usepackage{textcomp}
\usepackage[english]{babel}
\usepackage{amsmath, amssymb}
\usepackage{listings}
\usepackage[left=1in,right=1in,top=1in,bottom=1in]{geometry}

%figure support
\usepackage{import}
\usepackage{xifthen}
\pdfminorversion=7
\usepackage{pdfpages}
\usepackage{transparent}
\newcommand{\incfig}[1]{%
	\def\svgwidth{\columnwidth}
	\import{./figures/}{#1.pdf_tex}
}
\graphicspath{ {./figures/} }
\pdfsuppresswarningpagegroup=1

\begin{document}
	\title{Test 2 Review}
	\author{Brandon Thompson}
	\maketitle

	\begin{enumerate}
		\item Convert between machine code and assembly.\\
			\verb|0x2237FFF1|\\
			Convert to binary $\to$ \verb+0010|0010|0011|0111|1111|1111|1111|0001+\\
			separate into field values \verb+001000|10001|10111|1111 1111 1111 0001+\\
			get machine code from field values: op = 8, rs = 17, rt = 23, imm = -15\\
			Instruction is: \verb|addi $s7, $s1, -15|\\
			\\\verb|0x02F34022|\\
			Convert to binary $\to$ \verb+0000|0010|1111|0011|0100|0000|0010|0010+\\
			Separate into field values \verb+000000|10111|10011|01000|00000|100010+\\
			Get machine code from field values: op = 0, rs =23, rt = 19, rd = 8, shamt = 0, funct = 34\\
			Instruction: \verb|sub $t0,$s7,$s3|\\
			\\Instruction \verb|lw $t2, 32($0)|\\
			Field values: op = 35, rs = 0, rt = 10, imm = 32\\
			Convert to binary field values: \verb+100011|00000|01010|0000 0000 0010 0000+\\
			Separate to bytes: \verb+1000|1100|0000|1010|0000|0000|0010|0000+\\
			Convert to hexadecimal: \verb|0x8C0A0020|\\

		\item Convert between high level and assembly.
			\begin{itemize}
				\item Loops \verb+beq blt slt j jal jr+
				\item Arrays: character, integer, \ldots
				\item Functions: 4+ arguments, 64 bit returns, stack pointer, Recursion\\
					Convert from C to MIPS assembly:\\
					C Code
					\begin{verbatim}
						int array[1000];
						int i;

						for (i = 0; i < 1000; i++)
						    array[i] = array[i] + 8;
					\end{verbatim}
					MIPS Assembly Code
					\begin{verbatim}
						# $s0 = array base address, $s1 = i
						lui $s0, 0x2300
						ori $s0, $s0, 0xF000
						addi $s1, $0, 0
						addi $t2, $0, 1000

						loop:
						slt $t0, $s1, $t2
						.
						.
						.
					\end{verbatim}
					\\
					C Code
					\begin{verbatim}
						int factorial(int n) {
						  if (n <= 1)
						    return 1;
						  else
						    return (n * factorial(n-1));
					\end{verbatim}
					MIPS Code
					\begin{verbatim}
						factorial:
						addi $sp, $sp, -8    # make room
						sw $a0, 4($sp)       # stores $a0
						sw $ra, 0($sp)       # stores $ra
						addi $t0, $0, 2
						slt $t0, $a0, $t0    # n <= 1
						.
						.
						.
					\end{verbatim}

			\end{itemize}
		\item Characters use \verb+lb+ (load byte) instruction instead of
			\verb+lw+ (load word) instruction.\\
			C Code
			\begin{verbatim}
				int array[5];
				array[0] = array[0] = 2;
				array[1] = array[1] = 2;
			\end{verbatim}
			MIPS Code
			\begin{verbatim}
				lui #s0, 0x1234
				ori $s0, $a0, 0x8000

				lw $t1, 0($a0)
				sll $t1, $t1, 1
				sw $t1, 0($a0)
				.
				.
				.
			\end{verbatim}
		\item Addressing Modes
			\begin{itemize}
				\item Base Addressing\\
					Address of operand is: \verb+base address+
				\item PC-Relative Addressing
				\item Pseudo-direct Addressing\\
					Take 32-bit address and make it a 26-bit address by
					removeing the first 4 and last 2 bits
			\end{itemize}
	\end{enumerate}
	
\end{document}
