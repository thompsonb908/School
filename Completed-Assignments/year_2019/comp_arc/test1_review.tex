\documentclass[a4paper]{article}

\usepackage[utf8]{inputenc}
\usepackage[T1]{fontenc}
\usepackage{textcomp}
\usepackage[english]{babel}
\usepackage{amsmath, amssymb}


%figure support
\usepackage{import}
\usepackage{xifthen}
\pdfminorversion=7
\usepackage{pdfpages}
\usepackage{transparent}
\newcommand{\incfig}[1]{%
	\def\svgwidth{\columnwidth}
	\import{./figures/}{#1.pdf_tex}
}

\pdfsuppresswarningpagegroup=1

\begin{document}
\author{Brandon Thompson}
\title{Midterm Review}
\maketitle
\medskip
	\begin{description}
		\item[Heirarcy] is a top down approach
		\item[Modularity] interfaces
		\item[Regularity] compartmentalizatino
		\item[Digital Abstraction] A to D conversion
		\item[Boolean Logic] Min term is sum of products, Max term is product of sums.
		\item[Number formats] Binary $101011_2 \to 2B_{16} \to 53_8$
			\begin{equation}
				\log_2(8)=3 \text{ use every 3 bits}, \log_2(16)=4 \text{ use every 4 bits}
			\end{equation}
		\item[Kilobyte to kilobit] $\frac{kB}{4}=kb$  $2^{32} = 2^{30}+2^2 = 4Gb$ 
		\item[Micro] arangement of \ldots 
		\item[Assembly Language] Human readable format.
		\item[Machine Code] binary
		\item[Principles of Computer Architecture] smaller is faster, instruction set limited, limited
			operantds, instruction format limited. Simplicity favors regularity, common format(R,J,I),
			simple ISA (Load store, registers). Good design demands good comprimises, decoding in hardware.
			Make the common case fast, decoding in hardware (instruction format is similar), smaller
			instruction set.
		\item[Data Storage] 1 word = 32 bits, byte addressable (each word needs 4 locations).
		\item[little Endinas] lowest byte = lowest address
		\item[big endian] highest byte = lowest address\\
			helps with data transfer, MIPS string operations.
		\item[branch instructions] 
		\item[instructino types] I,J,R
		\item[Difference between RISK and SiSC] R = MIPS, S = x86
		\item[memory Heirarchy]
		\item[Convery assembly to machine] assembly > block > binary (page 309 in textbook)
	\end{description}
	
	\*
	Convert  addi s_7,s_1,-15
	\begin{center}
		
	\end{center}

\end{document}
