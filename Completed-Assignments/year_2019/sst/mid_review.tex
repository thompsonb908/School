\documentclass[a4paper]{article}

\usepackage[utf8]{inputenc}
\usepackage[T1]{fontenc}
\usepackage{textcomp}
\usepackage[english]{babel}
\usepackage{amsmath, amssymb}


%figure support
\usepackage{import}
\usepackage{xifthen}
\pdfminorversion=7
\usepackage{pdfpages}
\usepackage{transparent}
\newcommand{\incfig}[1]{%
	\def\svgwidth{\columnwidth}
	\import{./figures/}{#1.pdf_tex}
}

\pdfsuppresswarningpagegroup=1

\topmargin -.5in
\textheight 9in
\oddsidemargin -.125in
\evensidemargin -.125in
\textwidth 6.5in

\begin{document}
\title{Software Security Testing Midterm Review}
\author{Brandon Thompson}
\maketitle

	\begin{enumerate}
		\item CIA triad
			\begin{itemize}
				\item Confidentiality
				\item Integrity
				\item Availability
			\end{itemize}
		\item Software assurance definition
			\begin{itemize}
				\item Definition: The level of confidence that Software is free
					from vulnerabilities and functions in the intended manner.
			\end{itemize}
		\item 4 goals of software assurance 
			\begin{itemize}
				\item Trustworthy: Ensure no exploitable vulnerabilities or
					malicious logic exists in software.
				\item Dependability: Ensure the software, when executed,
					functions as intended.
				\item Survivability: Rugged and resilient
					\begin{itemize}
						\item If compromised, damage will be
							minimum.
						\item Will recover quickly to an acceptable capacity.
					\end{itemize}
				\item Conformance: Ensure Processes and products conform to requirements,
					standards, and procedures.
			\end{itemize}
		\item Computer security terminologies
			\begin{description}
				\item[Adversary (threat agent)] - An entity that attacks, or is a threat
					to, a system.
				\item[Attack] - An assault on system security that derives from an
					intelligent threat; a deliberate attempt to evade security
					services and violate security policy of a system.
				\item[Countermeasure] - An action, device, procedure, or technique
					that reduces a threat, vulnerability or attack.
					\begin{itemize}
						\item By eliminating or preventing it (prevent)
						\item By minimizing the harm it can cause (recover)
						\item By discovering and reporting it so that
							corrective action can be taken (detect)
					\end{itemize}
				\item[Threat] - A potential for violation of security, which exists
					when there is a circumstance, capability, action, or event
					that could breach security and cause harm.
				\item[Vulnerability] - Flaw or weakens in a system's design, implementation,
					or operation and management that could be exploited to violate the
					system's security policy
					\begin{itemize}
						\item Can be corrupted (loss of integrity)
						\item Can become leaky (loss of confidentiality)
						\item Can become unavailable (loss of availability)
					\end{itemize}
				\item[Risk] - An expectation of loss expressed as the probability that
					a particular threat will exploit a particular vulnerability
					with a harmful result.
					\begin{itemize}
						\item Low: limited adverse effect.
						\item Moderate: serious adverse effect.
						\item High: severe or catastrophic adverse effect.
					\end{itemize}
				\item[Security Policy] - A set of rules an practices that specify how a
					system or organization provides security services to protect
					sensitive and critical system resources.
				\item[System Resource (Asset)] - Data; a service provided by the system,
					a system capability; an item of system equipment; a facility that
					houses system operations and equipment.
					\begin{itemize}
						\item Hardware
						\item Software
						\item Data
						\item Communication facilities and networks.
					\end{itemize}
			\end{description}
		\item Types of General attacks
			\begin{description}
				\item[Active attack] is a network exploit in which a hacker attempts
					to make changes to data on the target.
				\item[Passive attack] is a network attack in which a system is
					monitored/scanned for open ports and vulnerabilities to gain
					information about the target.
				\item[Inside attack] is a malicious attack performed on a network or
					computer system by a person with authorized system access.
				\item[Outside attack] is initiated from outside the perimeter, by
					an unauthorized or illegitimate user of the system.
			\end{description}
		\item Types of specific attacks
			\begin{itemize}
				\item Social Engineering Attacks
					\begin{description}
						\item[Organization penetration] is tricking people at
							work into giving access to company resources.
						\item[Phishing] creating a malicious web site and 
							making it look like some other company's.
						\item[Spam] User clicks on email to read, email
							can install malware.
						\item[Spoofing] Change the ''From'' address in messages.
						\item[Man in the middle] unauthorized user requests
							or modifies messages between two parties.
					\end{description}
				\item Attacks against software
					\begin{description}
						\item[Cross-site scripting (XSS):] embed JS functions
							into HTML data element, and redisplayed on the
							web page as hyperlink. Once clicked, users will
							be directed to other websites without knowing.
						\item[Buffer overflows:] While writing data to a buffer,
							overruns the buffer's boundary and overwrites
							adjacent memory location.
						\item[SQL code injection:] Attack on DB web server that
							allows SQL statements to come in the application
							undetected.
						\item[Time/Logic bombs:] execute malicious code based
							on certain time or event.
						\item[Back door:] Bypass the application's security
							mechanism and uses the application resources
							to view or steal information.
					\end{description}
				\item Attacks against the supporting infrastructure
					\begin{description}
						\item[Denial of service (DOS):] Consume shared
							resources and compromise the ability of
							authorized users to access/use those resources.
						\item[Virus:] a program/code that replicates by being
							copied. A virus attaches itself to and becomes
							part of another program.
						\item[Worm:] A standalone malware computer program that
							replicates itself in order to spread to other
							computers. Often, it uses a computer network to
							spread itself.
						\item[Trojan:] Provide remote access to a system through
							a back door/open port.
						\item[Spyware:] software installed on a machine that
							secretly gathers information about user activity.
						\item[Adware:] a program that is unknowingly installed on
							the PC and produces ads while executing. Many
							adware come with spyware included.
					\end{description}
				\item Physical attacks
					
			\end{itemize}
		\item How to ensure quality/security in the cube
			\begin{itemize}
				\item Know the enemy
					\begin{itemize}
						\item Know weak areas of the application and where attackers
							are most likely going to attack first.
						\item Know who would want to attack your software and why.
						\item Know what types of resources would be needed by attackers
							such as tools, privileges, and time slots.
						\item Know how to build countermeasures.
					\end{itemize}
				\item Prevent social engineering
					\begin{itemize}
						\item Verify callers
						\item Only give information to identified people.
						\item Share information on a need-to-know basis.
						\item watch out for shoulder surfing.
					\end{itemize}
				\item Clean up the clutter
					\begin{itemize}
						\item Do not keep sticky notes with passwords on them
							in or around your desk.
						\item Delete old and unnecessary hard and soft documents.
					\end{itemize}
				\item Stay current
					\begin{itemize}
						\item Keep informed of the latest software attacks.
					\end{itemize}
			\end{itemize}
		\item Principles and concepts of secure software
			\begin{itemize}
				\item Secure the weakest link
					\begin{itemize}
						\item The weakest part of the system will most likely be
							attacked first.
					\end{itemize}
				\item Defense in depth
					\begin{itemize}
						\item Multiple layers of different types of protection
							provide substantially better protection.
						\item Goal is to limit access to certain features of
							the application.
					\end{itemize}
				\item Fail securely
					\begin{itemize}
						\item What happens when the system goes down.
						\item Address error-handling issues appropriately.
						\item Degrade peacefully.
					\end{itemize}
				\item Least privilege
					\begin{itemize}
						\item Give users the least amount of privilege required
							to perform the use case.
						\item Applications that need access to other system
							resources; grant only what is needed.
					\end{itemize}
				\item Keep it simple
					\begin{itemize}
						\item Keep security simple and keep the application simple.
						\item Keep the design simple.
						\item Keep the database and code as simple as possible.
					\end{itemize}
				\item Secrets are not kept
					\begin{itemize}
						\item Binary code is not secure code.
						\item Do not share passwords.
						\item Do not place hard-coded values in code.
						\item Place secrets in external resources e.g., DB.
						\item Remove comments that reveal secrets.
					\end{itemize}
				\item Complete mediation
					\begin{itemize}
						\item Access to every object must be checked for authority.
					\end{itemize}
				\item Separation of privilege
					\begin{itemize}
						\item System should not grant permission based on single
							condition.
						\item Company checks over \$75,000 need to be signed
							by two officers.
					\end{itemize}
			\end{itemize}
		\item Principles and concepts of quality software
			\begin{itemize}
				\item Understandability
					\begin{itemize}
						\item Variables given meaningful names.
						\item Logic and loops coded an easy to follow way.
						\item If a person does not understand the programming
							language, they should be able to follow the logic.
					\end{itemize}
				\item Flexibility and reusability
					\begin{itemize}
						\item Can the code be modified easily without affecting
							a lot of other modules and programs?
						\item Can the code be reused or other purposes?
						\item Repeatedly used blocks of code should be made into
							subroutines.
					\end{itemize}
				\item Readability and capability
					\begin{itemize}
						\item Is code so long that a person gets lost trying to
							follow the execution path?
						\item Are inputs validated before use?
					\end{itemize}
				\item Maintainability and testability
				\item Usability and reliability
					\begin{itemize}
						\item Is there adequate online help?
						\item Is a user manual provided?
						\item Are meaningful error messages provided?
						\item Will the software perform when needed?
						\item Is exception handling provided?
					\end{itemize}
			\end{itemize}
		\item Difference between authorization and authentication
			\begin{description}
				\item[Authorization:] Ensuring that the user has the appropriate role
					and privilege to view data.
				\item[Authentication:] Ensuring that the user is who he or she claims
					to be and that the data comes from the appropriate place.
			\end{description}
		\item Devise misuse cases
			\begin{enumerate}
				\item 
			\end{enumerate}
		\item Definition of assets
			\begin{description}
				\item[Asset:] Anything of value to the stakeholders.
			\end{description}
		\item ATM case study
	\end{enumerate}
\end{document}
