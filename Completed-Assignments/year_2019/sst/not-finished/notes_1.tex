\documentclass[a4paper]{article}

\usepackage[utf8]{inputenc}
\usepackage[T1]{fontenc}
\usepackage{textcomp}
\usepackage[english]{babel}
\usepackage{amsmath, amssymb}


%figure support
\usepackage{import}
\usepackage{xifthen}
\pdfminorversion=7
\usepackage{pdfpages}
\usepackage{transparent}
\newcommand{\incfig}[1]{%
	\def\svgwidth{\columnwidth}
	\import{./figures/}{#1.pdf_tex}
}

\pdfsuppresswarningpagegroup=1

\begin{document}
	\title{Security Requirements}
	\maketitle
	\medskip
	The software development life cycle:
	\begin{enumerate}
		\item Requirement analysis
			\begin{itemize}
				\item[--] Security Requirements
			\end{itemize}
		\item Design
			\begin{itemize}
				\item[--] Misuse Cases / Vulnerability Mapping
			\end{itemize}
		\item Construction
			\begin{itemize}
				\item[--] Secure Coding Practices
			\end{itemize}
		\item Testing
			\begin{itemize}
				\item[--] Penetration Testing
			\end{itemize}
		\item Installation
			\begin{itemize}
				\item[--] Final Security Review
			\end{itemize}
		\item Maintenance
			\begin{itemize}
				\item[--] Periodic Security Review and Update
			\end{itemize}
	\end{enumerate}
	\\
	The earlier security is considered, the more likely it is to be implemented well.
	\section{Gathering Requirements}
	\begin{description}
		\item[Requirement] is an outcome for the proposed system, something that it must perform
			or a quality it must have.
		\item[Functional Requirement] is something that the system must do.
		\item[Nonfunctional Requirement] is a quality or constraint for the system; must be upheld.
		\item[Security Requirement] is an associated protection
	\end{description}

	\section{Functional and Nonfunctional Security}
	Asking and answering the following questions will create a well-written requirement:
	\begin{enumerate}
		\item Why should this be part of the system?
		\item What are the constraints on this requirement?
		\item What are the dependencies on this requirement?
		\item %fill
	\end{enumerate}

	\section{Security Requirement}
	\begin{description}
		\item[Fail case:] what will happen if the requirement is not fulfilled during operation
		\item[Consequnce of failure:]
		\item[Associated risk:]
	\end{description}
	\begin{itemize}
		\item What are the exceptions to the normal case for this requirement?
		\item that sensitive info is included?
		\item What are the consequences if the conditions are violated?
		\item What happens if this requirement is intentionally violated?
	\end{itemize}

	%fill Example 1

	%fill Example 2

	%fill Example 3
	
	\section{Validation}
	\begin{description}
		\item[Validation:] is the process of making sure the right system is being built.
		\item[Validation Testing:] asserting that the needs of the system and stakeholders
			are being med with the requirements.
		\item[Tradeoff-analysis:]
	\end{description}

	%Discussion questions

	%difference between a security requirement and a security metric
	
\end{document}
