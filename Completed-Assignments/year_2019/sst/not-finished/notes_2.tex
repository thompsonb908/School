\documentclass[a4paper]{article}

\usepackage[utf8]{inputenc}
\usepackage[T1]{fontenc}
\usepackage{textcomp}
\usepackage[english]{babel}
\usepackage{amsmath, amssymb}


%figure support
\usepackage{import}
\usepackage{xifthen}
\pdfminorversion=7
\usepackage{pdfpages}
\usepackage{transparent}
\newcommand{\incfig}[1]{%
	\def\svgwidth{\columnwidth}
	\import{./figures/}{#1.pdf_tex}
}

\pdfsuppresswarningpagegroup=1

\begin{document}
	\title{Chapter 5: Software requirements}
	\maketitle
	Requirements are usually given by the BA's (Business analysts) and told ''ask if you have any questions''
	\begin{itemize}
		\item Understand given requirements.

		\item Analyze similar products or services.
		\item Interpret with your peers.
	\end{itemize}
	\\
	%high and medium level requirements
	gathering security requirements becoming just as important as functionality.
\\
	identify assets in the requirements.
	\begin{description}
		\item[Asset:] anything of value to the stakeholders.
	\end{description}
	\begin{itemize}
		\item When reviewing use case diagram, look for assets.
	\end{itemize}

	\section{Devise Misuse Cases}
	In software, every function that the application's software provides is also an open
	opportunity\ldots\\
	This is a hard thing to do because:
	\begin{itemize}
		\item little methodologies in this area
		\item \ldots
	\end{itemize}
	%every use case will have a corresponding misuse case (the relevant attack for the use case)
	In order to devise the misuse cases you must: 
	\begin{itemize}
		\item know your enemy
		\item carry out relevant attacks
		\item design countermeasures for relevant attacks
	\end{itemize}
	\\
	Misuse case actors could be an outside person, an inside person or an automated machine.\\
	for carrying out the relevant attacks, use the list of attacks previously (XSS, SQL Injection).\\
	Every countermeasure is a new requirement in the requirements list. Implementing a log
	in counter will prevent a brute force attack.% add other examples
\end{document}
