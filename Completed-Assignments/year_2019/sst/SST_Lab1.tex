\documentclass[a4paper]{article}

\usepackage{tikz}
\usepackage{graphicx}
\graphicspath{ {/home/brandon/Documents/images/} }

\newcommand{\numpy}{{\tt numpy}}	% tt font for numpy

\topmargin -.5in
\textheight 9in
\oddsidemargin -.25in
\evensidemargin -.25in
\textwidth 7in

\begin{document}
	\author{Brandon Thompson: 5517}
	\title{Lab 1: CEN4088.01 Due ??/??/2019}
	\maketitle
	
	\begin{center}
		\includegraphics[width=.8\textwidth]{one}\\
		\captionof{Figure 1: Result of \texttt{Nmap} OS scan for TargetVulnerable01.}
		\label{fig:1.1.3}
	\end{center}

	\begin{center}
		\includegraphics[width=0.8\textwidth]{two}\\
		\captionof{Figure 2: Details of threats to TargetWindows04.}
		\label{fig:1.2.9}
	\end{center}
	
	\begin{center}
		\includegraphics[width=0.8\textwidth]{three}\\
		\captionof{Figure 3: Details of threats to TargetWindows05.}
		\label{fig:1.2.21}
	\end{center}
	
	\begin{center}
		\includegraphics[width=0.8\textwidth]{four}\\
		\captionof{Figure 4: \texttt{Nmap] scan of reduced attack surface for TargetVulnerable01.}
		\label{fig:1.3.25}
	\end{center}
	\\\ \\
	Figure 1 shows six open ports in TargetVulnerable01 including port 445 which has a 
	major known vulnerability involved. Section 3 enables applies rules to and enables the Windows Firewall.
	This process reduces the number of open ports to one, the remote access port needed for the lab,
	effectively securing the system.\\

	\begin{center}
		\includegraphics[width=0.8\textwidth]{five}\\
		\captionof{Figure 5: Inbound rule for TargetWindows04.}
		\label{fig:1.4.6}
	\end{center}
	
	\begin{center}
		\includegraphics[width=0.8\textwidth]{six}\\
		\captionof{Figure 6: \texttt{Nmap} scan of reduced attack surface for TargetWindows04.}
		\label{fig:1.4.24}
	\end{center}
	Figure 5 shows the result of removing all inbound rules from the Windows Firewall except the
	remote desktop rule needed for the lab. Figure 6 shows result of \texttt{Nmap} command after
	applying rules (figure 5), which has reduced the number of open ports on the machine to one.
	By following the steps in section 4, we have reduced the possible vulnerability on TargetWindows04.\\
	
	\begin{center}
		\includegraphics[width=0.8\textwidth]{seven}\\
		\captionof{Figure 7: Result of \texttt{tcpdump} using decoy IP addresses.}
		\label{fig:2.1.8}
	\end{center}

	\begin{center}
		\includegraphics[width=0.8\textwidth]{eight}\\
		\captionof{Figure 8: Result of \texttt{Nmap} OS scan of TargetVulnerable01}
		\label{fig:2.1.11}
	\end{center}

	\begin{center}
		\includegraphics[width=0.8\textwidth]{nine}\\
		\captionof{Figure 9: Threat details of TargetWindows04.}
		\label{fig:2.2.8}
	\end{center}
	
	

\end{document}
