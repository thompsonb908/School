\documentclass[a4paper]{article}

\usepackage{indentfirst}
\usepackage[utf8]{inputenc}
\usepackage[T1]{fontenc}
\usepackage{textcomp}
\usepackage[english]{babel}
\usepackage{amsmath, amssymb}
\usepackage{url}

%figure support
\usepackage{import}
\usepackage{xifthen}
\pdfminorversion=7
\usepackage{pdfpages}
\usepackage{transparent}
\newcommand{\incfig}[1]{%
	\def\svgwidth{\columnwidth}
	\import{./figures/}{#1.pdf_tex}
}

\pdfsuppresswarningpagegroup=1

\begin{document}
	\title{Lab 3: CEN4048 Due xx/xx/19}
	\author{Brandon Thompson 5517}
	\maketitle
	\medskip
	\section{Section 1 Images}
	%images up to 1.3

	\section{Section 1 Research Report}
	\subsection{Executive Summary}
	
	The websites \url{x128bit.com}, \url{iskytap.com} and \url{cloudparadox.com} are
	managed by the same people because the registered administrator is the same for all
	three. An attacker could use this because weaknesses in one could be used in another,
	\url{x128bit.com} and \url{iskytap.com} are less secure than \url{cloudparadox.com}.

	Amazon being a large e-commerce company will be difficult to attack because they put
	so much effort into defending against it and verifying transactions. That being said
	because of the number of employees in the company they are more susceptible to social
	engineering attacks.

	\subsection{Methodology}
	The technical research involved the use of the Sam Spade tool to provide domain information
	from multiple sources about the site in question. The Sam Spade tool can also ping the
	site. The technical research also utilized the \texttt{nslookup} tool from the command
	prompt. \texttt{Nslookup} is used for DNS queries to provide associated IP addresses
	for a domain. The last tool used in this section is the \texttt{tracert} (traceroute) tool.
	Traceroute gives a list of names and IP addresses of all intermediate systems between the
	target and caller. This can provide additional attack points or the geographic location
	of the target system.
	
	\subsection{Technical Research Results}
	
	\subsubsection{\url{x128bit.com}}
	The Sam Spade tool provided the email address (\texttt{shulbert@securitycentric.net}) and phone number
	(+1 (925) 292-4309) of the registered administrator.Using the ping tool of Sam Spade provides 10 pings
	to the IP address \texttt{208.91.197.27}. The \texttt{nslookup} tool showed that there
	were no additional IP addresses associated. Traceroute gave 14 intermediary servers between
	the virtual machine and the target.
	
	\subsubsection{\url{iskytap.com}}
	\url{Iskytap.com} has the same information as the previous \url{x128bit.com}.	
	
	\subsubsection{\url{cloudparadox.com}}
	Sam Spade tool was unable to ping \url{cloudparadox.com} at IP address \texttt{50.225.131.227}.
	\texttt{Nslookup} provided no alternative IP addresses, traceroute timed out after jumping
	the $17^{\text{th}}$  time at \texttt{162.151.79.166}.

	\subsection{Public Domain Research Results}
	\label{PDR1}
	The organization that i decided to target is Amazon because they are a large e-commerce
	site and I use them often. The domain name is \url{amazon.com} and they use the same URL
	for their online sales. They physical address is listed as 410 Terry Avenue North Seattle,
	Washington 98109-5210, but they are planning to expand to another building soon. There are 10
	officers of the company, the CEO being Jeffrey Bezos, with 7 senior vice presidents and 2
	vice presidents. Amazon employs over 600,000 people, 15,000 of those are corporate employees
	in 40 different locations. Amazon also owns more than 40 subsidiaries including \url{audible.com},
	\url{A9.com}, Goodreads, Ring and \url{twitch.com}.

	\subsection{Findings and Conclusions}
	\url{X128bit.com} and \url{iskytap.com} could be vulnerable to attackers because we were able to
	ping the server multiple times and follow a traceroute to the home system. \url{Cloudparadox.com}
	is more secure because it does not allow pings and we could not follow the trace back. Amazon,
	being such a large company with almost all business going through the internet, probably has a 
	lot of security in place, but the large amount of employees means they are susceptible to social
	engineering attacks.
	
	\subsection{Avenues of Future Research}
	Additional research into the companies would include how secure the physical building is for
	social engineering attacks or physical attacks. If i was planning a hack into these companies
	I would ask questions like how would they defend against \underline{\ \ \ \ \ \ \ \ \ \ \ \ \ }
	or what is the best method to gain access to the system.
	
	\section{Section 2 Research Report}
	\subsection{Executive Summary}
	
	\subsection{Methodology}

	\subsection{Technical Research Results}
	
	\subsection{Public Domain Research Results}
	Target organization if Amazon, all info is the same as Section 1 Research Report \ref{PDR1	Target organization if Amazon, all info is the same as Section 1 Research Report \ref{PDR1}}
	\subsection{Findings and Conclusions}

	\subsection{Avenues of Further Research}


\end{document}
