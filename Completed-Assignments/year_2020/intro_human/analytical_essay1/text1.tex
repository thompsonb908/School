In {\it Player Piano}, Kurt Vonnegut utilizes irony as an important, humorous role in the story to take away from the depressing reality of the ''utopia'' that Paul lives in.
In the novel, machines are the backbone of the economy, production is determined by a supercomputer called EPICAC and filtered down into decreasing levels of managers who run the machines. 
Engineers and managers started to believe the propaganda that was said about them to keep them from participating in the war, \autocite[62]{kurt1} which leads them to believe that they are an elite class.
While the machines work to make life easier for people by producing goods virtually eliminating manufacturing error, they have also eliminated the common man's reason for living.

EPICAC is featured in another of Vonnegut's works and was inspired by ENIAC, the worlds first electronic general-purpose computer.
It is interesting to note that EPICAC shares a similar pronunciation with a commonly used expectorant and emetic of the time, ipecac.
Vonnegut is hinting to how the development of ENIAC will effect society.
In small doses, it will have a positive cleaning effect, and in high doses, a violent forceful expulsion, like the one depicted in this book. 

Most of the population is unhappy with their lot in life.
Take Private First Class Elmo Hackets in Chapter 7 \autocite[47--49]{kurt1}, his inner monologue is very angry with the world, the military and being told what to do.
He describes his daily routine of cleaning his area and making sure that everything is in its proper place.
This notion of people not having anything to do but mundane tasks is seen in many other places throughout the book, namely when the Reconstruction and Reclamation Corps was filling a pothole on the bridge.
Paul observes that "About 40 men, leaning on crowbars, picks and shovels, blocked the way," with only one man in the center doing work. \autocite[25]{kurt1}

Paul's life is filled with irony, the first of which is the fact that Paul has the highest paying job at Ilium, yet he still drives an old beat up car with a missing headlight.
This is because he is imitating Finnerty's distaste for cleanliness while also trying to occasionally intermingle with the Homestead population. Paul married Anita because she said that she was pregnant with Paul's child, but she turns out to be barren.
Paul continually has to remind himself that he does love Anita, until she starts her affair with Shepherd, where Paul is convinced of his love, but Anita is not.
This can be seen in the punctuation of the 'you' whenever they say, "I love \textit{you}."
Paul gets drunk one night and yells, to nobody in particular, that, ''We must meet in the middle of the bridge!'' 
He decides that, "he would be the only one interested in the expedition," and "the only one who didn't feel strongly about which bank he was on." \autocite[73]{kurt1}
When Paul gets assigned he tent-mate for the Meadows competition, it turns out to be the man that is competing for the Pittsburgh job he has been trying to get.

These points show that Paul's character and beliefs are conflicted to his core.
He does not believe in any one side 100\% and is constantly put into a position to choose what he thinks is right, but always manages to toe the line between pro-machine and anti-automation, practically until the very end.
Probably to Paul's benefit, nobody seems to care about what side Paul is on and continue with their plans with him regardless.
This is showcased, with humor, on \autocite[142]{kurt1} where it is determined by his superiors that Paul will be a spy planted in the Ghost Shirt society.
In response to this, Paul tries to quit to which the others laugh and joke, "Keep that up and you'll fool the hell out of them."

Bud is a major character whose ironic nature is critical to the plot, he is known for tinkering, making gadgets to perform some tasks, and is often heard saying how easy it would be to accomplish a task with a cheap machine.
He started as a manager of the petroleum terminal for Ilium Works, in his free time, designing a gadget that did his job (a little too well he states), effectively making his entire job class obsolete (72 people).
Bud was upset, as many were, with being replaced, and sent to the Reeks and Wrecks.
In the end, Bud, and others, repair the Orange-O machine that they had just destroyed.
Even when the machine is functioning again, a woman says, "But the light behind the Orange-O sign didn't light up, supposed to."
Bud and the man that fixed Paul's car started working on repairing the light and describes the man as, "desperately unhappy then. Now he was proud and smiling because his hands were busy doing what the liked to do best, Paul supposed - replacing men like himself with machines." \autocite[204]{kurt1}

This brings to light the main point of the book, that people will complain about the way things are and want to go back to the way things were, even if they are not necessarily good for them.
During Paul's final toast before turning himself in, he starts to say "To a better world," but knowing that the people of Ilium are "already eager to recreate the same old nightmare" decides to toast as his fellow leaders do, not to a better society, but to "the record" for knowing that it could be done. \autocite[205--206]{kurt1}
 
On a separate plot line, the Shah, a visiting spiritual leader from another country continually compares the general population to slaves, because in his culture there are only the elite and slaves, which mimics the introduction that states, Ilium New York is divided into three parts, the managers and engineers, the machines, and the Homestead.
The purpose of the separate plot line is to show the difference between an insider of the system and one that is looking at it from a distance.
To engineers and managers, they are living in a utopia where they provide for the majority and are rewarded for it, where the majority of people are engineered out of jobs and forced to live dreary monotonous lives.
As an outsider, the Shah laughs at the Army and the Reeks and Recks because they are treated as slaves, and act like slaves, they just do not know they were slaves.
In the end the Shah shows some sympathy to a woman who decides to whore herself out in order to support her husband who had written an anti-machine book and been barred from publishing.

In the foreword, Kurt warns that this "is not a book about what is, but a book about what could be." \autocite[11]{kurt1}
It is clear that Vonnegut thinks negatively about what the increase of automation will do to society if followed blindly and the prospect that "a step backward, after making a wrong turn, is a step in the right direction" \autocite[190]{kurt1} is one we may need to impose in the future.