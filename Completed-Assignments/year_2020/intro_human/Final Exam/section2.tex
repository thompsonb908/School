Technological normalization is how people grow accustomed to a new technology and is very apparent in Dave Eggers \textit{The Circle}.
Kalden, or Ty is quoted with saying \textquote{some of the things we did, I just\----I did just to see if anyone would actually use them.} \autocite[262]{eggers1}
This shows that even though the creator did not believe that the idea was good for society, he wanted to see just how far people were willing to go.
The theme of Technological normalization is constant throughout the book, starting with cameras meant to watch for human rights abuses and ending with trying to read peoples thoughts.
The final words speak volumes on this \textquote{The world deserved nothing less and would not wait.} \autocite[268]{eggers1}

\textit{Oryx and Crake} by Margaret Atwood shows aspects of \textit{homo faber} through Crake creating his 'Crakers.'
Crake believed that humanity was faulty and by creating his 'superior' people to replace them he was taking the fate of humanity into his own hands.

\textit{The Circle} by Dave Eggers is heavily infused with panopticism.
The releasing of SeeChange cameras saturated the society fairly quickly with watchers.
Later Mai's SoulSearch program meant to find people that had previously evaded detection further exposed the population to panopticism.
Overall, the Circle was using the panoptic society to gather information about its users for monetization, similar to what large tech companies are trying to do now.