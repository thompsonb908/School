\textquote{My God, May though. It's heaven.} \autocite[1]{eggers1} These first few sentences are how May, the main character, describe her first look at the campus of The Circle.
The Circle is the best tech company to ever exist, they hold a monopoly for social media, advertising and pretty much anything else.
Because of their success, the Circle seemingly has an unlimited supply of funds to decorate their campus and supply their employees with an endless stream of social events.
Dave Eggers uses the art and buildings at the company, as well as the activities of May, as symbols for the dangers of technology and transparency presented in \textit{The Circle.}

Eggers describes the campus of the Circle in detail in the beginning of the novel, allowing the reader to feel the scale and lavishness of the campus.
Boasting everything from sporting areas and a doggy daycare to shopping centers and a cafeteria made completely of glass, the campus is a testimony to the benefits of technology.
To showcase the amount of exposure that the Circle offers, musicians, comedians, and writers book the stage in the cafeteria a year in advance to perform for the Circlers.
May is astounded by the budget to get so book so many famous people, to which Annie replies \textquote{Oh god, we don't \textit{pay} them.} \autocite[14]{eggers1}
This shows how important the Circle is for publicity, that famous artist will come from all over the world, for free, to sing in their cafeteria and gather exposure.

As May falls into her daily tasks at the Circle, more layers to her work are pushed upon her.
Originally, she starts with two screens, and is flattered that the Circle would supply her, a new employee, with so much on her first day.
Slowly amassing 11 screens, two of which are attached to her body via wrist bracelets, as she becomes more impactful at the company.
The screens show the different layers that are forced on her under the guise of \textquote{fun.}
Gina describes the first three of May's screen as the first being CE responsibilities that are paramount, the second screen shows messages from anyone directly supervising your work, prioritizing these second, the third screen is for social media, Inner- and OuterCircle, \textquote{these messages aren't, like, superfluous.
They're just as important as any other messages, but are prioritized third.
And sometimes they are urgent.} \autocite[57]{eggers1} 
And the fourth priority being her OuterCircle, which is \textquote{just as important as anything else.} \autocite[57]{eggers1}
The PartiRank is a ranking for the most socially active employees, described as being \textquote{just a fun way to  see how your participation manifests itself} \autocite[58]{eggers1} even though it is integrated into her work routine.
The screens show May's progression through the ranks of the Circle, as well as the information overload she is being subjected to, when one of her ranks rises, the others fall, and she has to work through the night to correct them.

Eggers consistently describes May's positions of number of Zings, PartiRank, health metrics, Conversion Rate and Retail Raw rankings with great accuracy.
These numbers are analogous to May's position within the company, better numbers show how well May is doing socially and publicity wise.
May often loses sleep about the state of her rankings before she goes transparent, spending hours to boost her PartiRank.
Knowing the exact numbers had no impact on the plot.
Instead, Eggers uses these numbers to describe the obsession the Circle has with data.
To justify storing all of this data, the Circle invents metrics to track their employees, giving them yet another layer of complexity to be conscious about.

One of the decorations that the Circle has commissioned is a fourteen foot high plexiglass sculpture titled \textquote{\textit{Reaching Through for the Good of Humankind}} \autocite[190]{eggers1} by a Chinese artist.
The sculpture depicts a hand reaching through a screen.
Circlers find it obvious of the Circles good deeds with the way they have been able to reach through the screen and connect with their users.
Some of the interpretations believe that \textquote{He's trying to say that we need more ways to reach through the screen} \autocite[190]{eggers1} and \textquote{the screen is a barrier and we are transcending it.} \autocite[190]{eggers1}
The in-text artist and Eggers most likely mean for the sculpture to be a work of satire, showing that for humanity to save itself, we need to \textquote{reach through} the screen and free ourselves from technology.
Eggers is able to tie in the sculpture's content with its material, the transparent plexiglass a symbol for the Circle, showing that it is the technology accompanied by transparency that will cause humanities downfall.

Later, while May is broadcasting Stenton's aquarium to her viewers, she showcases some of the creatures that Stenton has recovered from the Marianas Trench, a seahorse and its babies, an octopus, and a shark.
The main attraction is the feeding of the shark, like many attractions at the Circle, and because of the environment it was found, the shark is translucent.
The caretaker feeds the shark a lobster and a sea turtle, to document how the shark is able to digest crustaceans and large animals.
The shark is known for its rapid digestion process, swallowing its prey whole, letting its stomach enzymes do the work, and disposing of nondigestible particulate extremely quickly.
The shark is used as an analogy to Stenton himself, who's appearance is compared to a shark, and the Circle itself.
The shark is translucent, an aspect that the Circle is pushing on many of its employees and users.
Despite this, it devours defenseless creatures, eating, digesting and disposing of unwanted parts without regard, foreshadowing the future if the Circle is completed.

The Circle was founded on simplifying the lives of its users.
By looking into May's life in the Circle we can see that the further she went into the circle, the more complicated her life got.
The themes of technology, privacy and transparency are the highlight of the novel and Eggers uses many tools to illuminate them to the reader.
Symbolism in \textit{The Circle} is very apparent through the use of art and the activities of Circlers to describe their inability to see what is wrong with how far they have taken technology.
Eggers utilizes the symbolism to provide the reader a sense of foreboding about the technology presented in the novel in the hopes that this kind of future can be avoided.