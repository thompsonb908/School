In {\it Player Piano}, Kurt Vonnegut utilizes irony as important, humorous role in the story.
In the novel, machines are the backbone of the economy, and thus the one that run them are considered elites.
While the machines work to make life easier for by producing goods and basically getting rid of manufacturing error, they have also gotten rid of the point of living.

The majority of the population is unhappy with their lot in life.
Take Private First Class Elmo Hackets in Chapter 7 \autocite{chapter7}(47-49), his inner monologue is very angry with the world and the military and being told what to do.
This notion is seen in many other places throughout the book where people in the Reconstruction and Reclamation Corps will give their information to Paul for recommendations for better job.

Paul's life has some ironic points as well. Paul married Anita because she said that she was pregnant with Paul's child but she turns out to be barren.
Paul has the highest paying job at Ilium yet he still drives an old beat up car with a missing headlight. This could be because he is imitating Finnertys distaste for cleanliness.

Another major point of irony was when Bud, a manager of the petroleum terminal for Ilium designed a gadget that did his job.
This made his entire job class obsolete (72 people) and because his personality score was low on his test, he was unable to acquire another type of job.

On a separate plot line, the Shah, a visiting spiritual leader from another country keeps comparing the general population to slaves, because in his culture there are only the elite and slaves, which mimics the introduction that states, Ilium New York is divided into three parts, the managers and engineers, the machines, and the Homestead.
The purpose of the separate plot line is to show the difference between an insider of the system and one that is looking at it from a distance.
To engineers and managers, they are living in a utopia where they provide for the majority and are rewarded for it.
Where the majority of people are engineered out of jobs
As an outsider, the Shah laughs at the Army and the Reeks and Recks because they are slaves, they just did not know it.

In the foreword, Kurt warns that this "is not a book about what is, but a book about what could be." \autocite{foreword}(11)
It is clear the Vonnegut thinks that the rise of industrialization is a bad thing if followed blindly.