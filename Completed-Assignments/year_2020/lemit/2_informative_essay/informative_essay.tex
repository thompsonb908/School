\documentclass[12pt]{article}
\usepackage[top=1in, bottom=1in, left=1in, right=1in]{geometry}
%\usepackage[margin=1in]{geometry}
\usepackage[onehalfspacing]{setspace}
%\usepackage[doublespacing]{setspace}
%\usepackage{amsmath, amssymb, amsthm}
\usepackage{enumerate, enumitem}
%\usepackage{fancyhdr, graphicx, proof, comment, multicol}
\usepackage[none]{hyphenat} % This command prevents hyphenation of words
%\binoppenalty=\maxdimen % This command and the next prevent in-line equation breaks
%\relpenalty=\maxdimen

\usepackage{hyperref}

%    Good website with common symbols
% http://www.artofproblemsolving.com/wiki/index.php/LaTeX%3ASymbols
%    How to change enumeration using enumitem package
% http://tex.stackexchange.com/questions/129951/enumerate-tag-using-the-alphabet-instead-of-numbers
%    Quick post on headers
% http://timmurphy.org/2010/08/07/headers-and-footers-in-latex-using-fancyhdr/
%    Info on alignat
% http://tex.stackexchange.com/questions/229799/align-words-next-to-the-numbering
% http://tex.stackexchange.com/questions/43102/how-to-subtract-two-equations
%    Text align left-center-right
% http://tex.stackexchange.com/questions/55472/how-to-make-text-aligned-left-center-right-in-the-same-line

%\usepackage{microtype} % Modifies spacing between letters and words
%\linespread{1.1} 

\usepackage[
backend=biber,
style=mla,
showmedium=false
]{biblatex}
\addbibresource{informative_essay.bib}

\begin{document}
% This is the Header
% Make sure you update this information!!!!
\noindent
\textbf{IDS 2144.01} \hfill \textbf{Brandon Thompson} \\
\normalsize Prof. LeFrancois \hfill Due Date: 6/1/2020 \\

% This is where you name your homework
\begin{center}
\textbf{Informative Essay}
\end{center}

	Between 2004 and 2009 almost 40,000 mass layoffs occurred in the US, affecting over
	7 million workers. \autocite{sucher} Technology is advancing at a rapid rate, with
	advancements in Artificial Intelligence (AI) and robotics, your job could soon be
	automated. This paper will cover the ethical implications of replacing large amounts
	of workers with robotic systems. There are many issues with replacing workers with
	machines. What are the economic impacts of laying off hundreds of workers. Should 
	employers be allowed to replace capable workers with machines? What should be done
	with the workers who have been replaced? I used to be an intern for a factory that
	assembled lithium-ion battery packs for medical and military level equipment. Every
	year that I was interning there there was another machine that was faster and could
	do more than the last year. While i was at the factory, they were in the process of
	expanding their operation so they did not need to lay anybody off, but I got to
	witness firsthand how automation has changed the manufacturing process.
	
	Automation of a job is not the same as the automation of a task, Bessen states in his
	paper that a group of researchers  evaluated 70 occupations and deemed 37 of them
	''fully automatable,'' and predicted that in the future, half of all jobs are
	susceptible to complete automation. However, none of the 37 jobs listed have been
	completely automated so far. \autocite{bessen} Automation is usually not the main
	cause of job loss, older occupations like telegraph operators were made obsolete
	because of the advancement of technology, not because the process was automated.
	Automation is the sole purpose of the decline of elevator operators however.
	Soon another industry might be at risk of being completely automated, truck drivers.
	Autonomous vehicles are expected to revolutionize road travel soon.
	Truck driving provides many jobs for low-skill workers with 93\% of the work force
	having less than a bachelors degree.\autocite{algernon}
	
	Automating jobs could be a good thing for the workers, people in the welding profession
	can develop eye, nose and throat irritation as well as pulmonary infections, heart
	disease, and lung and throat cancer as a product of working with hazardous materials.
	\autocite{pham} Replacing these workers with a machine would not only increase production
	for the company, it will reduce health issues caused by welding. Pham also states that
	just because workers that retain their jobs might not see conditions improve, because
	the pace of the machines is much faster than the humans, the machines must work at
	the pace of the humans, which means that the people are overworked compared to a
	machine that can do many more cycles in less time than a human.

	The time frame for the complete automation of jobs is not academically agreed upon.
	Some \autocite{ford} argue that the cognitive capability of automation carries an
	actual threat of massive job destruction over the coming decades. And \autocite{frey}
	Oxford economists predict that 47\% of total US employment is at risk of being taken
	away by automation. While a MIT Technology Review survey of dozens of global economic
	and technology experts shows no consensus on the number of jobs lost or when.
	\autocite{winick}

	All in all, automation is inevitable, but the automation of jobs does not have to be
	a bad thing, there are many cases where automation has increased the number of jobs
	in the field effected. This just brings to light how fragile the economy is and how
	careful we need to be when deploying new technologies in mass quantities.

	\newpage
	\printbibliography
\end{document}
