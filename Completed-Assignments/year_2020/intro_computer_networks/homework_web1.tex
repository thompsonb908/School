\documentclass[12pt]{article}
\usepackage[top=1in, bottom=1in, left=1in, right=1in]{geometry}
%\usepackage[margin=1in]{geometry}
\usepackage[onehalfspacing]{setspace}
%\usepackage[doublespacing]{setspace}
\usepackage{amsmath, amssymb, amsthm}
\usepackage{enumerate, enumitem}
\usepackage{fancyhdr, graphicx, proof, comment, multicol}
\usepackage[none]{hyphenat} % This command prevents hyphenation of words
\binoppenalty=\maxdimen % This command and the next prevent in-line equation breaks
\relpenalty=\maxdimen
%    Good website with common symbols
% http://www.artofproblemsolving.com/wiki/index.php/LaTeX%3ASymbols
%    How to change enumeration using enumitem package
% http://tex.stackexchange.com/questions/129951/enumerate-tag-using-the-alphabet-instead-of-numbers
%    Quick post on headers
% http://timmurphy.org/2010/08/07/headers-and-footers-in-latex-using-fancyhdr/
%    Info on alignat
% http://tex.stackexchange.com/questions/229799/align-words-next-to-the-numbering
% http://tex.stackexchange.com/questions/43102/how-to-subtract-two-equations
%    Text align left-center-right
% http://tex.stackexchange.com/questions/55472/how-to-make-text-aligned-left-center-right-in-the-same-line
\usepackage{microtype} % Modifies spacing between letters and words
\usepackage{mathpazo} % Modifies font. Optional package.
\usepackage{mdframed} % Required for boxed problems.
\usepackage{parskip} % Left justifies new paragraphs.
\linespread{1.1} 


%figure support
\usepackage{import}
\usepackage{xifthen}
\pdfminorversion=7
\usepackage{pdfpages}
\usepackage{transparent}
\newcommand{\incfig}[1]{%
	\def\svgwidth{\columnwidth}
	\import{./figures/}{#1.pdf_tex}
}
\graphicspath{ {./figures/} }
\pdfsuppresswarningpagegroup=1

\newenvironment{problem}[1]
{\begin{mdframed}[linewidth=0.8pt]
        \textsc{Problem #1:}

}
    {\end{mdframed}}

\newenvironment{solution}
    {\textsc{Solution:}\\}
    {\newpage}% puts a new page after the solution
    
\newenvironment{statement}[1]
{\begin{mdframed}[linewidth=0.6pt]
        \textsc{Statement #1:}

}
    {\end{mdframed}}

%\newenvironment{prf}
 %   {\textsc{Proof:}\\}
 %   {\newpage}% puts a new page after the solution

\begin{document}
% This is the Header
% Make sure you update this information!!!!
\noindent
\textbf{CNT 3004C-01 Intro to Computer Networks} \hfill \textbf{Brandon Thompson} \\
\normalsize Prof. Ding \hfill Due Date: 2/24/20 \\

% This is where you name your homework
\begin{center}
\textbf{Web 1}
\end{center}
	Obtain the HTTP/1.1 specification (RFC 2616). Answer the following questions:
	\begin{problem}{\#1}
		Explain the mechanism used for signaling between the client and server
		to indicate that a persistent connection is being closed. Can the
		client, the server, or both signal the close of a connection?
	\end{problem}
	\begin{solution}
		\begin{itemize}
			\item Mechanism used for signaling persistent connection to close
				is a connection-token of ''close'' in the connection
				header sent in the request that request becomes the last
				one for the connection (section 8.1.2.1 of RFC 2616).
			\item Both the client and server can signal the closing of a
				connection by using the Connection header field 
				(section 8.1.2 of RFC 2616).
		\end{itemize}
	\end{solution}

	\begin{problem}{\#2}
		Can a client open three or more simultaneous connections with a given server?
	\end{problem}
	\begin{solution}
		Clients should limit the number of simultaneous connections that they
		maintain to a given server. A single-user client should not maintain
		more than 2 connections with any server or proxy. A proxy should use
		up to $2*N$ where $ N$ is the number of simultaneous active users
		(section 8.1.4 RFC 2616).

		Assuming that the client in question is a single-user client, no,
		they would only be able to open 2 connections to a server.
	\end{solution}

	\begin{problem}{\#3}
		Either a server or a client may close a transport connection between them if
		either one detects a connection has been idle for some time. Is it possible
		that one side starts closing a connection while the other side is transmitting
		data via this connection? Explain.
	\end{problem}
	\begin{solution}
		A client, server or proxy may close the transport connection at any time. For
		For example, a client might have started to send a new request at the same time
		that the server decides to close the ''idle'' connection. From server's point of
		view, the connection is being closed because it was idle, but from the client's
		point of view there is a request in progress (section 8.1.4 RFC 2616).

		Yes, a connection could be closed on one end but not on another. RFC also states
		that ''clients, servers, and proxies must be able to recover from asynchronous
		close events.''
	\end{solution}
\end{document}
