\documentclass[12pt]{article}
\usepackage[top=1in, bottom=1in, left=1in, right=1in]{geometry}
%\usepackage[margin=1in]{geometry}
\usepackage[onehalfspacing]{setspace}
%\usepackage[doublespacing]{setspace}
\usepackage{amsmath, amssymb, amsthm}
\usepackage{enumerate, enumitem}
\usepackage{fancyhdr, graphicx, proof, comment, multicol}
\usepackage[none]{hyphenat} % This command prevents hyphenation of words
\binoppenalty=\maxdimen % This command and the next prevent in-line equation breaks
\relpenalty=\maxdimen
%    Good website with common symbols
% http://www.artofproblemsolving.com/wiki/index.php/LaTeX%3ASymbols
%    How to change enumeration using enumitem package
% http://tex.stackexchange.com/questions/129951/enumerate-tag-using-the-alphabet-instead-of-numbers
%    Quick post on headers
% http://timmurphy.org/2010/08/07/headers-and-footers-in-latex-using-fancyhdr/
%    Info on alignat
% http://tex.stackexchange.com/questions/229799/align-words-next-to-the-numbering
% http://tex.stackexchange.com/questions/43102/how-to-subtract-two-equations
%    Text align left-center-right
% http://tex.stackexchange.com/questions/55472/how-to-make-text-aligned-left-center-right-in-the-same-line
\usepackage{microtype} % Modifies spacing between letters and words
\usepackage{mathpazo} % Modifies font. Optional package.
\usepackage{mdframed} % Required for boxed problems.
\usepackage{parskip} % Left justifies new paragraphs.
\linespread{1.1} 


%figure support
\usepackage{import}
\usepackage{xifthen}
\pdfminorversion=7
\usepackage{pdfpages}
\usepackage{transparent}
\newcommand{\incfig}[1]{%
	\def\svgwidth{\columnwidth}
	\import{./figures/}{#1.pdf_tex}
}
\graphicspath{ {./figures/} }
\pdfsuppresswarningpagegroup=1

\newenvironment{problem}[1]
{\begin{mdframed}[linewidth=0.8pt]
        \textsc{Problem #1:}

}
    {\end{mdframed}}

\newenvironment{solution}
    {\textsc{Solution:}\\}
    {\newpage}% puts a new page after the solution
    
\newenvironment{statement}[1]
{\begin{mdframed}[linewidth=0.6pt]
        \textsc{Statement #1:}

}
    {\end{mdframed}}

%\newenvironment{prf}
 %   {\textsc{Proof:}\}
 %   {\newpage}% puts a new page after the solution

\begin{document}
% This is the Header
% Make sure you update this information!!!!
\noindent
\textbf{CIS4367.01} \hfill \textbf{Brandon Thompson} \\
\normalsize Prof. Elibol \hfill Due Date: 1/27/20 \\

% This is where you name your homework
\begin{center}
\textbf{Homework 4}
\end{center}
	\begin{problem}{1}
		Briefly summarize what you know about the field of
		cryptography.
	\end{problem}
	\begin{solution}
		The purpose of cryptography is to prevent unwanted people from from gaining
		access to information that should be private, or changing information without
		the owners knowledge.\\ In the early days all cryptographic algorithms relied
		on the alphabet, by transposing the letters in a certain patter or swapping
		the location of letters many times. These can easily be broken by frequency
		analysis. \\
		Later came rotor machines that sped up the encryption process
		during WWII by using rotors to create a faster, more secure method of encryption
		that could not be broken using frequency analysis. \\
		Current cryptographic algorithms rely on prime factorization for security
		because multiplying numbers is much easier than finding the correct
		factors from a very large  number.
	\end{solution}
	
	\begin{problem}{2}
		List seven questions or things you don't know about
		cryptography. For each question you list, indicate why
		it might be important to know the answer.
	\end{problem}
	\begin{solution}
		\begin{enumerate}
			\item How does the encryption process actually happen?\\
				It would be important to know the steps and an in
				depth example for debugging purposes or ensuring
				proper functionality.
			\item How does one go about attacking a ciphertext?\\
				Knowing how to break into things is important in
				understanding how to secure things.
			\item What are some mathematical algorithms used in encryption?\\
				Understanding the current standards, and why they are used
				is important for implementing encryption in a system.
			\item How does one stop an attacker from changing a message
				and just generating a new hash?\\
				This is important because an attacker could change all of
				your messages and hashed and the recipient would never know.
			\item How does elliptic curve cryptography work?\\
				Elliptic curve cryptography is being implemented in more and
				more things recently.
			\item Why does changing a key size make things more difficult to
				break into?\\
				The correlation between key size and computation speed is
				important to find a balance between the two.
			\item How will common cryptographic algorithms hold up to quantum
				computing in the next few years.\\
				Quantum computing is starting to become more prevalent, and
				boasts much better processing power than current systems.
				Will quantum computers force new algorithms to be developed.
		\end{enumerate}
	\end{solution}

	\begin{problem}{3}
		Describe situations in your life when you might need to
		use encryption or secret codes. How important would it be
		to have enough understating of the subject of cryptography
		to assess the strength of the code used?
	\end{problem}
	\begin{solution}
		If i needed to store private information on a public server I
		would like to know what algorithms are strong enough to secure
		my data. How to verify that nobody has modified my data. How to
		securely send data to someone else.\\
		With the proper understanding of cryptographic codes, I could
		ensure that my data is secure and it has not been tampered with.
	\end{solution}

	\begin{problem}{4}
		Prepare a one paragraph objective summary of the main ideas
		in this chapter.
	\end{problem}
	\begin{solution}
		Symmetric encryption uses a single key (secret key) to encrypt or decrypt,
		and requires both the sender and receiver to share the same key.
		Asymmetric encryption uses two keys (public / private key) to
		encrypt and decrypt information. MAC algorithms and hash functions
		are used to verify the validity of data by sending the message along with
		a hash value. If the message is altered en route, the receiver will
		know because the hash will not be the same. The combination of encryption
		and MAC algorithms provides very good security.
	\end{solution}
\end{document}
