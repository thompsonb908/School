\documentclass[12pt]{article}
\usepackage[top=1in, bottom=1in, left=1in, right=1in]{geometry}
%\usepackage[margin=1in]{geometry}
\usepackage[onehalfspacing]{setspace}
%\usepackage[doublespacing]{setspace}
\usepackage{amsmath, amssymb, amsthm}
\usepackage{enumerate, enumitem}
\usepackage{fancyhdr, graphicx, proof, comment, multicol}
\usepackage[none]{hyphenat} % This command prevents hyphenation of words
\binoppenalty=\maxdimen % This command and the next prevent in-line equation breaks
\relpenalty=\maxdimen
%    Good website with common symbols
% http://www.artofproblemsolving.com/wiki/index.php/LaTeX%3ASymbols
%    How to change enumeration using enumitem package
% http://tex.stackexchange.com/questions/129951/enumerate-tag-using-the-alphabet-instead-of-numbers
%    Quick post on headers
% http://timmurphy.org/2010/08/07/headers-and-footers-in-latex-using-fancyhdr/
%    Info on alignat
% http://tex.stackexchange.com/questions/229799/align-words-next-to-the-numbering
% http://tex.stackexchange.com/questions/43102/how-to-subtract-two-equations
%    Text align left-center-right
% http://tex.stackexchange.com/questions/55472/how-to-make-text-aligned-left-center-right-in-the-same-line
\usepackage{microtype} % Modifies spacing between letters and words
\usepackage{mathpazo} % Modifies font. Optional package.
\usepackage{mdframed} % Required for boxed problems.
\usepackage{parskip} % Left justifies new paragraphs.
\linespread{1.1} 


%figure support
\usepackage{import}
\usepackage{xifthen}
\pdfminorversion=7
\usepackage{pdfpages}
\usepackage{transparent}
\newcommand{\incfig}[1]{%
	\def\svgwidth{\columnwidth}
	\import{./figures/}{#1.pdf_tex}
}
\graphicspath{ {./figures/} }
\pdfsuppresswarningpagegroup=1

\newenvironment{problem}[1]
{\begin{mdframed}[linewidth=0.8pt]
        \textsc{Problem #1:}

}
    {\end{mdframed}}

\newenvironment{solution}
    {\textsc{Solution:}\\}
    {\newpage}% puts a new page after the solution
    
\newenvironment{statement}[1]
{\begin{mdframed}[linewidth=0.6pt]
        \textsc{Statement #1:}

}
    {\end{mdframed}}

%\newenvironment{prf}
 %   {\textsc{Proof:}\\}
 %   {\newpage}% puts a new page after the solution

\begin{document}
% This is the Header
% Make sure you update this information!!!!
\noindent
\textbf{CIS 4367} \hfill \textbf{Brandon Thompson} \\
\normalsize Prof. Elibol \hfill Due Date: 4/27/2020 \\

% This is where you name your homework
\begin{center}
\textbf{Homework 21}
\end{center}
	\begin{problem}{\#1}
		Summarize what you know at this point about firewalls. What points or questions would you like to discuss further?
	\end{problem}
	\begin{solution}
		Firewalls block network traffic according to rules that the administrator has defined. Traffic that is defined as bad is blocked, the packets do not progress past the firewall, good traffic is allowed to pass through. Firewalls can be placed in many locations, at the network edge, just before end nodes, and between secure and insecure networks. The placement and configuration of a firewall are critical in the safety of the network.
	\end{solution}

	\begin{problem}{\#2}
		Prepare a one-paragraph objective summary of the main ideas in this chapter (Chapter 9).
	\end{problem}
	\begin{solution}
		(See firewall summary above)\\
		Intrusion prevention systems add even more security on top of firewalls by monitoring traffic for abnormalities through protocol, statistical, or traffic anomalies, or pattern and stateful matching. When an anomaly is detected, the traffic is flagged with where it came from, where it was going to, and what it was trying to do to give the administrator(s) a way to document and defend against attacks in the future. 
	\end{solution}
\end{document}
