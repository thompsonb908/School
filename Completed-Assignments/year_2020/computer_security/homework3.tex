\documentclass[12pt]{article}
\usepackage[top=1in, bottom=1in, left=1in, right=1in]{geometry}
%\usepackage[margin=1in]{geometry}
\usepackage[onehalfspacing]{setspace}
%\usepackage[doublespacing]{setspace}
\usepackage{amsmath, amssymb, amsthm}
\usepackage{enumerate, enumitem}
\usepackage{fancyhdr, graphicx, proof, comment, multicol}
\usepackage[none]{hyphenat} % This command prevents hyphenation of words
\binoppenalty=\maxdimen % This command and the next prevent in-line equation breaks
\relpenalty=\maxdimen
%    Good website with common symbols
% http://www.artofproblemsolving.com/wiki/index.php/LaTeX%3ASymbols
%    How to change enumeration using enumitem package
% http://tex.stackexchange.com/questions/129951/enumerate-tag-using-the-alphabet-instead-of-numbers
%    Quick post on headers
% http://timmurphy.org/2010/08/07/headers-and-footers-in-latex-using-fancyhdr/
%    Info on alignat
% http://tex.stackexchange.com/questions/229799/align-words-next-to-the-numbering
% http://tex.stackexchange.com/questions/43102/how-to-subtract-two-equations
%    Text align left-center-right
% http://tex.stackexchange.com/questions/55472/how-to-make-text-aligned-left-center-right-in-the-same-line
\usepackage{microtype} % Modifies spacing between letters and words
\usepackage{mathpazo} % Modifies font. Optional package.
\usepackage{mdframed} % Required for boxed problems.
\usepackage{parskip} % Left justifies new paragraphs.
\linespread{1.1} 


%figure support
\usepackage{import}
\usepackage{xifthen}
\pdfminorversion=7
\usepackage{pdfpages}
\usepackage{transparent}
\newcommand{\incfig}[1]{%
	\def\svgwidth{\columnwidth}
	\import{./figures/}{#1.pdf_tex}
}
\graphicspath{ {./figures/} }
\pdfsuppresswarningpagegroup=1

\newenvironment{problem}[1]
{\begin{mdframed}[linewidth=0.8pt]
        \textsc{Problem #1:}

}
    {\end{mdframed}}

\newenvironment{solution}
    {\textsc{Solution:}\\}
    {\newpage}% puts a new page after the solution
    
\newenvironment{statement}[1]
{\begin{mdframed}[linewidth=0.6pt]
        \textsc{Statement #1:}

}
    {\end{mdframed}}

%\newenvironment{prf}
 %   {\textsc{Proof:}\}
 %   {\newpage}% puts a new page after the solution

\begin{document}
% This is the Header
% Make sure you update this information!!!!
\noindent
\textbf{CIS4367.01} \hfill \textbf{Brandon Thompson} \\
\normalsize Prof. Elibol \hfill Due Date: 1/27/20 \\

% This is where you name your homework
\begin{center}
\textbf{Homework 3}
\end{center}
	\begin{problem}{2.1}
		Suppose that someone suggests the following way to confirm that
		the two of you are both in possession of the same secret key.
		You create a random bit string the length of the key, and send
		the result over the channel. Your partner XORs the incoming block
		with the key, and sends it back. You check, and if what you 
		receive is your original random string, you have verified that
		your partner has the same secret key, yet neither of you has
		ever transmitted the key. Is there a flaw in this scheme?
	\end{problem}
	\begin{solution}
		Yes, because the random bit string and the XORed value are both
		transferred over the network, if someone was listening on the
		communication and got both of these values, the XOR of the two
		would be the secret key.
	\end{solution}

	\begin{problem}{2.2}
		This problem uses a real-world example of a symmetric cipher,
		from an old U.S. Special Forces manual (public domain). The
		document, filename \textit{Special Forces.pdf}, is available at
		\verb|box.com/CompSec3e|.
		\begin{enumerate}[label=\alph*]
			\item Using the two keys (memory words) \textit{cryptographic}
				and \textit{network security}, encrypt the following
				message:

				Be at the third pillar from the left outside the
				lyceum theatre tonight at seven. If you are distrustful
				bring two friends.

				Make reasonable assumptions about how to treat redundant
				letters and excess letters in the memory words and how to
				treat spaces and punctuation. Indicate what your assumptions
				are.\\
				\textit{Note:} The message is from the Sherlock Holmes novel
				\textit{The Sign of Four}.
			\item Decrypt the ciphertext. Show your work.
			\item Comment on when it would be appropriate to use this technique
				and what its advantages are.
		\end{enumerate}
	\end{problem}
	\begin{solution}
		\begin{center}
			\begin{tabular}{|c|c|c|c|c|c|c|c|c|c|}
				\hline
				2 & 8 & 10 & 7 & 9 & 6 & 3 & 1 & 4 & 5 \\ \hline
				C & R & Y  & P & T & O & G & A & H & I \\ \hline
				B & E & A  & T & T & H & E & T & H & I \\
				R & D & P  & I & L & L & A & R & F & R \\
				O & M & T  & H & E & L & E & F & T & O \\
				U & T & S  & I & D & E & T & H & E & L \\
				Y & C & E  & U & M & T & H & E & A & T \\
				R & E & T  & O & N & I & G & H & T & A \\
				T & S & E  & V & E & N & X & X & X & X \\
				I & F & Y  & O & U & A & R & E & D & I \\
				S & T & R  & U & S & T & F & U & L & B \\
				R & I & N  & G & T & W & O & F & R & I \\
				E & N & D  & S & X & X & X & X & X & X \\ \hline
			\end{tabular}
			\quad
			\begin{tabular}{|c|c|c|c|c|c|c|c|c|c|}
				\hline
				4 & 2 & 8 & 10 & 5 & 6 & 3 & 7 & 1 & 9 \\ \hline
				N & E & T & W  & O & R & K & S & C & U \\ \hline
				T & R & F & H  & E & H & X & E & U & F \\
				X & B & R & O  & U & Y & R & T & I & S \\
				R & E & E & A  & E & T & H & G & X & R \\
				F & O & X & H  & F & T & E & A & T & X \\
				D & L & R & X  & I & R & O & L & T & A \\
				X & I & B & I  & X & H & L & L & E & T \\
				I & N & A & T  & W & X & T & I & H & I \\
				U & O & V & O  & U & G & S & E & D & M \\
				T & C & E & S  & F & T & I & N & T & L \\
				E & D & M & N  & E & U & S & T & X & A \\
				P & T & S & E  & T & E & Y & R & N & D \\ \hline
			\end{tabular}
			\\
			\vspace{10mm}
			Ciphertext:\\
			UIXTT EHDTX NRBEO LINOC DTXRH ELOTS ISYTX RFDXI UTEPE UEFIX WUFET
			HYTTR HXGTU EETGA LLIEN TRFRE XRBAV EMSFS RXATI MLADH OAHXI TOSNE
		\end{center}

		Number of letters: 110\\
		Size of keyword: 10\\
		Number of rows: $\frac{110}{10} = 11$\\
		Write ciphertext into columns based on order defined by the second memory word.\\
		Transpose the rows into the columns defined by the first memory word.\\
		\begin{center}
			\begin{tabular}{|c|c|c|c|c|c|c|c|c|c|}
				\hline
				4 & 2 & 8 & 10 & 5 & 6 & 3 & 7 & 1 & 9 \\ \hline
				N & E & T & W  & O & R & K & S & C & U \\ \hline
				T & R & F & H  & E & H & X & E & U & F \\
				X & B & R & O  & U & Y & R & T & I & S \\
				R & E & E & A  & E & T & H & G & X & R \\
				F & O & X & H  & F & T & E & A & T & X \\
				D & L & R & X  & I & R & O & L & T & A \\
				X & I & B & I  & X & H & L & L & E & T \\
				I & N & A & T  & W & X & T & I & H & I \\
				U & O & V & O  & U & G & S & E & D & M \\
				T & C & E & S  & F & T & I & N & T & L \\
				E & D & M & N  & E & U & S & T & X & A \\
				P & T & S & E  & T & E & Y & R & N & D \\ \hline
			\end{tabular}
			\quad
			\begin{tabular}{|c|c|c|c|c|c|c|c|c|c|}
				\hline
				2 & 8 & 10 & 7 & 9 & 6 & 3 & 1 & 4 & 5 \\ \hline
				C & R & Y  & P & T & O & G & A & H & I \\ \hline
				B & E & A  & T & T & H & E & T & H & I \\
				R & D & P  & I & L & L & A & R & F & R \\
				O & M & T  & H & E & L & E & F & T & O \\
				U & T & S  & I & D & E & T & H & E & L \\
				Y & C & E  & U & M & T & H & E & A & T \\
				R & E & T  & O & N & I & G & H & T & A \\
				T & S & E  & V & E & N & X & X & X & X \\
				I & F & Y  & O & U & A & R & E & D & I \\
				S & T & R  & U & S & T & F & U & L & B \\
				R & I & N  & G & T & W & O & F & R & I \\
				E & N & D  & S & X & X & X & X & X & X \\ \hline
			\end{tabular}
			\\
			\vspace{10mm}
			Plaintext:\\
			Be at the third pillar from the left outside the lyceum theatre
			tonight at seven. If you are distrustful bring two friends.
		\end{center}
	\end{solution}

	\begin{problem}{2.5}
		In this problem we will compare the security services that are provided
		by digital signatures (DS) and message authentication codes (MAC). We
		assume that Oscar is able to observe all messages sent from Alice to
		Bob and vice versa. Oscar has no knowledge of any keys but the public
		one in case of DS. State whether and how (i) DS and (ii) MAC protect
		against each attack. The value auth(x) is computed with a DS or a MAC
		algorithm, respectively.
		\begin{enumerate}[label=\alph*]
			\item (Message integrity) Alice sends a message $x=$ ''Transfer
				\$1000 to Mark'' in the clear and also sends auth(x) to
				Bob. Oscar intercepts the message and replaces ''Mark''
				with ''Oscar.'' Will Bob detect this?
			\item (Replay) Alice sends a message  $x=$ ''Transfer \$1000 to Oscar''
				in the clear and also sends auth(x) to Bob. Oscar observes
				the message and signature and sends them 100 times to Bob.
				Will Bob detect this?
			\item (Sender authentication with cheating third party) Oscar claims
				that he sent some message  $x$ with  a valid auth(x) to Bob
				but Alice claims the same. Can Bob clear the questions in
				either case?
			\item (Authentication with Bob cheating) Bob claims that he received  
				a message $x$ with a valid signature auth(x) from Alice
				(e.g., "Transfer \$1000 from Alice to Bob") but Alice claims
				that she never sent it. Can Alice clear this question in
				either case?
		\end{enumerate}
	\end{problem}
	\begin{solution}
		\begin{enumerate}[label=\alph*]
			\item Will be detected by both (i) DS and (ii) MAC.
			\item Will \textbf{not} be detected by either (i) DS or (ii) MAC,
				unless a time-stamp is added.
			\item (i) DS: Bob has to verify the message with the public key from
				both. Only Alice's public key will give a valid verification.\\
				(ii) MAC: Bob has to receive the secret key from Alice and Bob
				(he already has these) and verify which message has the proper auth(x).
			\item (i) DS: Alice has to receive a copy of the message with the signature.
				Then Alice can show that message and signature can be verified with Bob's
				public key. Meaning Bob must have generated the message.
				(ii) MAC: No, Bob can say that Alice generated this message.
		\end{enumerate}
	\end{solution}

	\begin{problem}{2.6}
		Suppose $H\left( m \right) $ is a collision-resistant hash function that maps
		a message of arbitrary bit length into an $n$-bit hash value. Is it
		true that, for all messages  $x,x'$ with  $x \neq x'$, he have
		$H\left( x \right) \neq H\left( x' \right) $? Explain your answer.
	\end{problem}
	\begin{solution}
		Every hash function with more inputs than outputs will theoretically have collisions.
		The pigeonhole principle guarantees that some inputs will hash to the same outputs.
		\\
		Collision resistance does not mean that no collisions exist, only that they are harder
		to find.\\
		Birthday paradox states that if an function produces $N$ bits of output, an attacker
		who computes  $2^{\frac{N}{2}}$ hashes will find a pair that matches.
	\end{solution}
                
	\begin{problem}{2.8}
		Prior to the discovery of any specific public-key schemes, such as RSA,
		an existence proof was developed whose purpose was to demonstrate that
		public key encryption is possible in theory. Consider the functions
		$f_1\left( x_1 \right) =z_1$ ; $f_2\left( x_2,y_2 \right) = z_2$ ;
		$f_3\left( x_3,y_3 \right) = z_3$ where all values are integers with
		$1 \le x_i,y_i,z_i \le N$. Function $f_1$ can be represented by a vector
		\textbf{M1} of length $N$, in which the  $k$th entry is the value of $f_1\left( k \right)$
		. Similarly, $f_2$ and $f_3$ can be represented by $N \times N$ matrices
		\textbf{M2} and \textbf{M3}. The intent is to represent the
		encryption/decryption process by table look-ups for tables with very large
		values of $N$. Such tables would be impractically huge but could, in principle,
		be constructed. The scheme works as follows: Construct \textbf{M1} with a random
		permutation of all integers between $1$ and  $N$; that is, each integer appears
		exactly once in \textbf{M1}. Construct \textbf{M2} so that each new row contains
		a random permutation of the first $N$ integers. Finally, fill in \textbf{M3} to
		satisfy the following condition:
		\begin{equation*}
			f_3\left( f_2\left( f_1\left( k \right) ,p \right) ,k \right) = p \forall k,p
			\text{ with } 1\le k,p\le N
		\end{equation*}
		In words,
		\begin{enumerate}
			\item \textbf{M1} takes an input $k$ and produces an output $x$.
			\item \textbf{M2} takes inputs $x$ and $p$ giving output $z$.
			\item \textbf{M3} takes inputs $z$ and $k$ and produces $p$.
		\end{enumerate}
		The three tables, once constructed, are made public.
		\begin{enumerate}[label=\alph*]
			\item It should be clear that it is possible to construct \textbf{M3}
				to satisfy the preceding condition. As an example, fill
				in \textbf{M3} for the following simple case:
				\begin{center}
				M1 =\ \
				\begin{tabular}{|c|}
					\hline
					5 \\ \hline
					4 \\ \hline
					2 \\ \hline
					3 \\ \hline
					1 \\ \hline
				\end{tabular}
				\ \ M2 =\ \
				\begin{tabular}{|c|c|c|c|c|}
					\hline
					5 & 2 & 3 & 4 & 1 \\ \hline
					4 & 2 & 5 & 1 & 3 \\ \hline
					1 & 3 & 2 & 4 & 5 \\ \hline
					3 & 1 & 4 & 2 & 5 \\ \hline
					2 & 5 & 3 & 4 & 1 \\ \hline
				\end{tabular}
				\ \ M3 =\ \
				\begin{tabular}{|c|c|c|c|c|}
					\hline
					5 &\verb| | & \verb| | & \verb| | & \verb| | \\ \hline
					1 & & & & \\ \hline
					3 & & & & \\ \hline
					4 & & & & \\ \hline
					2 & & & & \\ \hline
				\end{tabular}
				\end{center}
				
				Convention: The $i$th element of \textbf{M1} corresponds to $k=i$. The
				$i$th row of \textbf{M2} corresponds to  $x=i$; the  $j$th column
				of \textbf{M2} corresponds to $p=j$. The  $i$th row of \textbf{M3}
				corresponds to  $z=i$; the  $j$th column of \textbf{M3} corresponds to
				$k=j$. We can look at this in another way. The  $i$th row of \textbf{M1}
				corresponds to the $i$th column of \textbf{M3}. The value of the
				entry in the  $i$th row selects a row of \textbf{M2}. The entries
				in the selected \textbf{M3} column are derived from the entries
				in the selected \textbf{M2} row. The first entry in the \textbf{M2} row
				dictates where the value 1 goes in the \textbf{M3} column. The second
				entry in the \textbf{M2} row dictates where the value 2 goes in the
				\textbf{M3} column, and so on.
			\item Describe the use of this set of tables to perform encryption and
				decryption between two users.
			\item Argue that this is a secure scheme.
		\end{enumerate}
	\end{problem}
	\begin{solution}
		\begin{enumerate}[label=\alph*]
			\item \hspace{2em} M3 =\ \
				\begin{tabular}{|c|c|c|c|c|}
					\hline
					5 & 2 & 4 & 1 & 5 \\ \hline
					1 & 4 & 2 & 3 & 2 \\ \hline
					3 & 1 & 5 & 2 & 3 \\ \hline
					4 & 3 & 1 & 4 & 4 \\ \hline
					2 & 5 & 3 & 5 & 1 \\ \hline
				\end{tabular}
			\item Bob selects random numbers to use as his private and public key.
				Sends the key to Alice. Alice encrypts the messages using Bobs
				public key, and the \textbf{M2} table. Bob can decrypt this using
				the \textbf{M1} and \textbf{M3} tables.
			\item The table used could be very large with random numbers as the inputs.
				This will give greater security and be infeasible to reverse 
				engineer the tables in a acceptable amount of time.
		\end{enumerate}
	\end{solution}
\end{document}
