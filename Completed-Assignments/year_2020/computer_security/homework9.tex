\documentclass[12pt]{article}
\usepackage[top=1in, bottom=1in, left=1in, right=1in]{geometry}
%\usepackage[margin=1in]{geometry}
\usepackage[onehalfspacing]{setspace}
%\usepackage[doublespacing]{setspace}
\usepackage{amsmath, amssymb, amsthm}
\usepackage{enumerate, enumitem}
\usepackage{fancyhdr, graphicx, proof, comment, multicol}
\usepackage[none]{hyphenat} % This command prevents hyphenation of words
\binoppenalty=\maxdimen % This command and the next prevent in-line equation breaks
\relpenalty=\maxdimen
%    Good website with common symbols
% http://www.artofproblemsolving.com/wiki/index.php/LaTeX%3ASymbols
%    How to change enumeration using enumitem package
% http://tex.stackexchange.com/questions/129951/enumerate-tag-using-the-alphabet-instead-of-numbers
%    Quick post on headers
% http://timmurphy.org/2010/08/07/headers-and-footers-in-latex-using-fancyhdr/
%    Info on alignat
% http://tex.stackexchange.com/questions/229799/align-words-next-to-the-numbering
% http://tex.stackexchange.com/questions/43102/how-to-subtract-two-equations
%    Text align left-center-right
% http://tex.stackexchange.com/questions/55472/how-to-make-text-aligned-left-center-right-in-the-same-line
\usepackage{microtype} % Modifies spacing between letters and words
\usepackage{mathpazo} % Modifies font. Optional package.
\usepackage{mdframed} % Required for boxed problems.
\usepackage{parskip} % Left justifies new paragraphs.
\linespread{1.1} 


%figure support
\usepackage{import}
\usepackage{xifthen}
\pdfminorversion=7
\usepackage{pdfpages}
\usepackage{transparent}
\newcommand{\incfig}[1]{%
	\def\svgwidth{\columnwidth}
	\import{./figures/}{#1.pdf_tex}
}
\graphicspath{ {./figures/} }
\pdfsuppresswarningpagegroup=1

\newenvironment{problem}[1]
{\begin{mdframed}[linewidth=0.8pt]
        \textsc{Problem #1:}

}
    {\end{mdframed}}

\newenvironment{solution}
    {\textsc{Solution:}\\}
    {\newpage}% puts a new page after the solution
    
\newenvironment{statement}[1]
{\begin{mdframed}[linewidth=0.6pt]
        \textsc{Statement #1:}

}
    {\end{mdframed}}

%\newenvironment{prf}
 %   {\textsc{Proof:}\\}
 %   {\newpage}% puts a new page after the solution

\begin{document}
% This is the Header
% Make sure you update this information!!!!
\noindent
\textbf{CIS4367.01} \hfill \textbf{Brandon Thompson} \\
\normalsize Prof. Elibol \hfill Due Date: 2/19/20 \\

% This is where you name your homework
\begin{center}
\textbf{Homework 9}
\end{center}
	\begin{problem}{\#1}
		Which do you think is harder to understand: DAC or RBAC? Explain.
	\end{problem}
	\begin{solution}
		I feel that the more difficult concept to grasp is RBAC because it is more 
		complicated than DAC. In DAC a user decides the permission of a file they own.
		In RBAC they system determines access based on if the user belongs to a group
		that has access to that file. RBAC gets more complicated with the different
		levels ( RBAC$_0$, RBAC$_1$, RBAC$_2$, RBAC$_3$ ). I am a little confused on the
		RBAC$_1$ and RBAC$_2$.
	\end{solution}

	\begin{problem}{\#2}
		Prepare a one-paragraph objective summary of the main ideas in this chapter (Chapter 4).
	\end{problem}
	\begin{solution}
		Access control is the main concern of computer security because if an attacker cannot
		gain access to the system the attack surface is severely limited. Methods of access
		control are Discretionary Access Control (DAC) where users manage their own access
		privileges, Mandatory Access Control (MAC) where the system determines who has access
		based on the security clearance of the user, Role-base Access Control (RBAC) where
		control is managed in groups, files can be access by multiple groups and users
		can be part of multiple groups, or Attribute-base Access Control (ABAC) where
		access is based on attributes of the user, the resource, and the environment.
	\end{solution}
\end{document}
