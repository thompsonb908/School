\documentclass[12pt]{article}
\usepackage[top=1in, bottom=1in, left=1in, right=1in]{geometry}
%\usepackage[margin=1in]{geometry}
\usepackage[onehalfspacing]{setspace}
%\usepackage[doublespacing]{setspace}
\usepackage{amsmath, amssymb, amsthm}
\usepackage{enumerate, enumitem}
\usepackage{fancyhdr, graphicx, proof, comment, multicol}
\usepackage[none]{hyphenat} % This command prevents hyphenation of words
\binoppenalty=\maxdimen % This command and the next prevent in-line equation breaks
\relpenalty=\maxdimen
%    Good website with common symbols
% http://www.artofproblemsolving.com/wiki/index.php/LaTeX%3ASymbols
%    How to change enumeration using enumitem package
% http://tex.stackexchange.com/questions/129951/enumerate-tag-using-the-alphabet-instead-of-numbers
%    Quick post on headers
% http://timmurphy.org/2010/08/07/headers-and-footers-in-latex-using-fancyhdr/
%    Info on alignat
% http://tex.stackexchange.com/questions/229799/align-words-next-to-the-numbering
% http://tex.stackexchange.com/questions/43102/how-to-subtract-two-equations
%    Text align left-center-right
% http://tex.stackexchange.com/questions/55472/how-to-make-text-aligned-left-center-right-in-the-same-line
\usepackage{microtype} % Modifies spacing between letters and words
\usepackage{mathpazo} % Modifies font. Optional package.
\usepackage{mdframed} % Required for boxed problems.
\usepackage{parskip} % Left justifies new paragraphs.
\linespread{1.1} 


%figure support
\usepackage{import}
\usepackage{xifthen}
\pdfminorversion=7
\usepackage{pdfpages}
\usepackage{transparent}
\newcommand{\incfig}[1]{%
	\def\svgwidth{\columnwidth}
	\import{./figures/}{#1.pdf_tex}
}
\graphicspath{ {./figures/} }
\pdfsuppresswarningpagegroup=1

\newenvironment{problem}[1]
{\begin{mdframed}[linewidth=0.8pt]
        \textsc{Problem #1:}

}
    {\end{mdframed}}

\newenvironment{solution}
    {\textsc{Solution:}\\}
    {\newpage}% puts a new page after the solution
    
\newenvironment{statement}[1]
{\begin{mdframed}[linewidth=0.6pt]
        \textsc{Statement #1:}

}
    {\end{mdframed}}

%\newenvironment{prf}
 %   {\textsc{Proof:}\\}
 %   {\newpage}% puts a new page after the solution

\begin{document}
% This is the Header
% Make sure you update this information!!!!
\noindent
\textbf{CIS4630} \hfill \textbf{Brandon Thompson} \\
\normalsize Prof. Elibol \hfill Due Date: -- \\

% This is where you name your homework
\begin{center}
\textbf{Quiz Review}
\end{center}
\begin{enumerate}[label=Q1.\arabic*]
		\item Define computer security.
			\begin{itemize}
				\item Protection of a system to preserve CIA of resources (hardware, software, firmware, information/data, telecommunications).
			\end{itemize}
		\item What is the difference between passive and active security threats.
			\begin{itemize}
				\item Passive attack monitor traffic and listen to data transmissions to gain information about the system.
				\item Active attacks try and gain access to system resources or modify data in transit.
			\end{itemize}
		\item List and define categories of passive and active network security attacks.
			\begin{itemize}
				\item Passive
					\begin{itemize}
						\item Eavesdropping: monitoring traffic and location of user.
						\item Collecting private data: Find contents of messages.
					\end{itemize}
				\item Active
					\begin{itemize}
						\item Replay: resend old packet to gain access.
						\item Forgery: Faking user access to gain entry to the system.
						\item Modification of Message: Modify transmitted messages to affect data integrity.
						\item Denial of service: interrupting service.
					\end{itemize}
			\end{itemize}

		\item List fundamental security design principals.
			\begin{itemize}
				\item Economy of mechanism: Design is simple.
				\item Fail safe defaults: verify access of subject.
				\item Open Design: Design of security system is easily available to all the viewers and users, by hiding credentials and restricting the access of the object within the system.
				\item Separation of privilege: restrict user access.
				\item Least Privilege: Only minimum amount of access is given to users.
				\item Least Common Mechanism: Restricts sharing of objects to reduce unnecessary communication between the users.
				\item Psychological Acceptance: security should not impede work flow while protecting system resources.
				\item Isolation: System resources should be separated from user resources.
				\item Encapsulation: Wrapping data and the program that works on that data and isolating from other user programs.
				\item Modularity: Design and implementation of security mechanisms or functions in a modular fashion.
				\item Layering: Building multiple layers of protection mechanisms.
				\item Least Astonishment: user friendly, user should understand security provided by the system.
			\end{itemize}
		\item Explain the difference between an attack surface and an attack tree.
			\begin{itemize}
				\item Attack surface: Reachable and exploitable vulnerabilities in a system.
				\item Attack Tree: Branching hierarchical data structure that represents a set of potential techniques for exploring security vulnerabilities.
			\end{itemize}
	\end{enumerate}
	\begin{enumerate}[label=Q2.\arabic*]
		\item What are the essential ingredients of a symmetric cipher?
			\begin{itemize}
				\item Plaintext
				\item Encryption Algorithm
				\item Secret Key
				\item Ciphertext
				\item Decryption Algorithm
			\end{itemize}
		\item How many keys are required for 2 people to communicate via a symmetric cipher?
			\begin{itemize}
				\item One key, symmetric encryption uses a single secret key for sender and receiver.
			\end{itemize}
		\item What are the two principal requirements for the secure use of symmetric encryption?
			\begin{itemize}
				\item Strong Encryption Algorithm: Opponent should not be able to decrypt ciphertext or discover key even if they are in possession of both ciphertext and plaintext.
				\item Sender and receiver must have copies of the secret key obtained and stored in a secure manner.
			\end{itemize}
		\item List 3 approaches to message authentication.
			\begin{itemize}
				\item Message Authentication Code.
				\item One way hash function: public key encryption and digital signature.
				\item Secure hash function.
			\end{itemize}
		\item What is a message authentication code?
			\begin{itemize}
				\item  Use of secret key to generate a small block of data, message authentication (MAX) appended to the message.
			\end{itemize}
		\item Briefly describe the three schemes illustrated in the figure below.
			%insert fig
			\begin{itemize}
				\item  Message plus MAC are transmitted, recipient calculates MAC for themselves and checks against the transmitted MAC.
			\end{itemize}
		\item What properties must a hash function have to be useful for message authentication.
			\begin{itemize}
				\item Takes variable length message and makes a fixed length hash value.
				\item Hard to get the original message from the hash.
				\item Easy to compute the hash from the message.
				\item Hard to find 2 hashes with 2 different messages.
			\end{itemize}
		\item What are the principal ingredients of a public-key cryptosystem?
			\begin{itemize}
				\item Plaintext
				\item Encryption Algorithm
				\item Public and Private key
				\item Ciphertext
				\item Decryption Algorithm
			\end{itemize}
		\item List and briefly define three uses of a public-key cryptosystem
			\begin{itemize}
				\item Encryption and decryption: sender encrypts a message with the recipients public key.
				\item Digital Signature: Sender signs a message with its private key.
				\item Key exchange: Two sides cooperate to exchange a session key.
			\end{itemize}
		\item What is the difference between a private key and a secret key?
			\begin{itemize}
				\item Secret key = symmetric encryption
				\item Private key = asymmetric encryption
			\end{itemize}
		\item What is a digital signature?
			\begin{itemize}
				\item Mechanism for authenticating a message. A method to ensure message authentication and integrity.
			\end{itemize}
		\item What is a public key certificate?
			\begin{itemize}
				\item Used to validate the public key of a user. (ID of user, public key of user, digital signature of the Certificate Authority).
			\end{itemize}
		\item How can public-key encryption be used to distribute a secret key for symmetric encryption?
			\begin{itemize}
				\item Digital Envelope (see figure below)
					\begin{itemize}
						\item Prepare a message.
						\item Generate a random symmetric key.
						\item Encrypt the message with the symmetric key.
						\item Encrypt symmetric key with receivers public key.
						\item Attach the encrypted symmetric key to the encrypted message and send to receiver.
					\end{itemize}
			\end{itemize}
			%figure for Digital Envelope
	\end{enumerate}
	\begin{enumerate}[label=Q3.\arabic*]
		\item In general terms, what are 4 means of authenticating a user's identity?
			\begin{itemize}
				\item \textit{Something the individual knows:} Password, PIN, answers to security questions.
				\item \textit{Something the individual possesses:} (Token authentication) Key-card, smart card, physical keys.
				\item \textit{Something the individual is (static biometric):} Fingerprint, retina, face.
				\item \textit{Something the individual does (dynamic biometric):} Voice, handwriting, typing rhythm.
			\end{itemize}
		\item List and briefly define the principal threats to the secrecy of passwords.
			\begin{itemize}
				\item \textit{Offline Dictionary Attack:} Attacker has access to system password file and compares hashes in the file to hashes of common passwords.
				\item \textit{Specific Account attack:} Brute force a specific account.
				\item \textit{Password guessing against a single user:} Attacker gains info about the account holder and password policies to better guess the password.
				\item \textit{Popular Password attack:} Use a popular password and check against multiple user ID's.
				\item \textit{Workstation Hijacking:} The attacker waits until a logged-in workstation is unattended.
				\item \textit{Exploiting user mistakes:} Strict policies make it more likely that a user will write down a password. Attacker tricks a user into giving up their password.
				\item \textit{Exploiting multiple passwords using electronic monitoring:} If a password is communicated across a network to log on to a remote system, it is vulnerable to eavesdropping.
			\end{itemize}
		\item List and briefly describe four common techniques for selecting or assigning passwords.
			\begin{itemize}
				\item User education: Show the user what a strong password is.
				\item Computer generated passwords: Long, random passwords are not easy to crack.
				\item Reactive password checking: System runs a password cracker and notifies user if it was able to crack their password.
				\item Proactive password checking: The user chooses his password based on rules given by the system.
			\end{itemize}
		\item Explain the difference between a simple memory card and a smart memory card.
			\begin{itemize}
				\item Memory Card: Stores but does not process data.
				\item Smart card: Has micro-processor, different types of memory, I/O ports, etc.
			\end{itemize}
		\item List and briefly describe the principal physical characteristics used for biometric identification.
			\begin{itemize}
				\item Facial characteristics
				\item Fingerprints
				\item Hand Geometry
				\item Retinal Pattern
				\item Iris
				\item Signature
				\item Voice
			\end{itemize}
		\item In the context of biometric user authentication explain the terms: enrollment, identification, and verification.
			\begin{itemize}
				\item Enrollment: Users have to be created to have access.
				\item Identification: The individual uses biometric sensor but presents no additional information.
				\item Verification: The user enters a PIN and also uses a biometric sensor.
			\end{itemize}
		\item Define the terms false match rate and false non-match rate, and explain the user threshold in relationship to these two rates.
			\begin{itemize}
				\item False match rate: Percent of invalid inputs that are incorrectly accepted.
				\item False non-match rate: Percent of valid inputs that are incorrectly rejected.
				\item By moving the threshold, the probabilities can be altered. An decrease in false match rate means an increase in false non-match rate, and vice versa.
			\end{itemize}
		\item Describe the general concept of a challenge-response protocol.
			\begin{itemize}
				\item Family of protocols in which one party presents a question and another party must provide a valid answer to be authenticated.
			\end{itemize}
	\end{enumerate}
	\begin{enumerate}[lable=Q4.\arabic*]
		\item What is the difference between authentication and authorization?
			\begin{itemize}
				\item \textit{Authentication:} Verifying who you are, confirming your own identity.
				\item \textit{Authorization:} Verify what you have access to, granting access to the system.
			\end{itemize}
		\item Define the difference between DAC and MAC.
			\begin{itemize}
				\item Discretionary Access Control (DAC), owner determines access rights.
				\item Mandatory Access Control (MAC), system provides access to the resource based on the clearance level of the user.
			\end{itemize}
		\item How does RBAC relate to DAC and MAC?
			\begin{itemize}
				\item  All three are not mutually exclusive and the access control mechanism can employ all policies to cover different resources.
				\item RBAC may use discretionary or mandatory mechanism for user roles.
				\item RBAC needs to have role access to an object, mandatory needs a tag.
			\end{itemize}
		\item List and define three classes of subject in an access control system:
			\begin{itemize}
				\item Owner
					\begin{itemize}
						\item 
					\end{itemize}
				\item Group
					\begin{itemize}
						\item 
					\end{itemize}
				\item World
					\begin{itemize}
						\item 
					\end{itemize}
			\end{itemize}
	\end{enumerate}
\end{document}
