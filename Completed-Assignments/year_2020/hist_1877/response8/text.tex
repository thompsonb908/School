Scientific racism stems from the idea of social darwinism, anthropology, and sociology where a race could be less fit for the environment and thus unfit to breed.
Whites believed that scientific methods could be used to classify races.
This gave a scientific basis for folk beliefs that blacks were of lesser station than whites.

An illustration from Harper's Weekly magazine depicted the differences between drawings of irish, english, and black males.
By drawing the irish and the black men with shorter foreheads and mouths pushed away from their faces, it showed that irish and black people had smaller brains and were closer in similarity to apes.
This was a way of showing that northern europeans were the height of evolution and should be the ruling race.

The united states military gave an intelligence test to all members of the military and mapped the results against race or position or country of origin.
Questions in these intelligence tests were very difficult for those that had not lived in the United States for a long time, or were not educated the same way.
This might have been to purposely skew the results in favor of norther whites.
Whatever the reason of these tests, the were not measuring intelligence directly, and were more of a trivia quiz.

Native born Americans saw immigrants as biologically inferior and were a threat to American cultural values.
This was mainly because of the amount of people that were immigrating and because immigrants were willing to work for lower wages.
This was threatening to Americans because it was seen as taking away jobs from Americans.