The Espionage Act was one of the ways that the Government suppressed dissent from the war effort.
What the Espionage Act did was prohibit spying, ''false statements'' that would impede the success of the US military, and most importantly, it gave the post office the power to stop the mailing of any publications that were deemed in violation.
This meant the publications critical of the Wilson administration and the war were not allowed to be mailed.
Because of this act the government was able to prosecute critics of the war.

Another way the government suppressed dissent was because the Sedition Act of 1918.
This act prohibited ''any disloyal, profane, scurrilous, or abusive language about the form of government of the United States.''
Being very vague, any person who criticized Wilson or the government could be prosecuted and sent to jail under this act.
More than 2,000 people were charged with violating the Act and more than 1,000 were convicted.
Despite being very against the concept of free speech, the Supreme Court determined that it was constitutional.

During World War 1, in order to support the war effort as much as they could, limited the free speech of citizens.
They did this to stop the spread of dissent even though today it would be extremely unconstitutional and would probably increase the dissent for the War and the government.