The presidential election of 1964 was one of the most clear ideological choices since 1936. The two candidates Linden Johnson and Barry Goldwater represented extremely different ideals.

Goldwaters followers wanted more aggressive conduct in the Cold War and that the liberal New Deal was a threat to \textquote{freedom}.
Goldwater voted against the Civil Rights Act arguing that it allowed the federal government to overreach its bounds.
The Johnson campaign attempted to stigmatize him as a dangerous radical that might cause civil war and a racist against the Civil Rights movement.

Johnson won the election with a 90\% electoral vote and 61\% of the popular vote.
While Johnson won almost every state, Goldwater was the first Republican to win the Deep South since the Reconstruction, setting up the south to realign to republican dominance.
Goldwater also challenged the liberal New Deal state.

After winning the election, LBJ launched the largest peacetime federal domestic program since the New Deal, calling it the Great Society.
The Great Society provided medical care for the poor and elderly through Medicaid and Medicare, and funded many educational, developmental, and cultural programs.
LBJ believed that poor people were poor because of an absence of skills and attitude opposed to inherit problems in the economy.

The election of LBJ extended the power and reach of the federal government, saw the return of the New Deal with the \textquote{freedom from want} slogan that included economic freedom in the definition of freedom.
