Typically when we think of settlers moving west we think of a individuals traveling alone to make a living farming. 
However, typically there would be a large family group, and sometimes neighbors traveling together and settling in the same general area.
Family groups, europeans, African Americans and ethnic communities are some of the types of people that traveled west.
More and more people started travel because of chain migration, people want to be around others that they know so they travel with them.
This is only possible because of railroads and government funding.

Railroad companies, governments and speculators distributed promotional literature describing what was needed to travel west, and how to start farming.
After traveling west, settlers had to acquire land, they could do this through the Homestead Act, where federal land would be given to any ''real'' settler.
More desireable land could be purchased from railroad companies, speculators, or from the government that was not included in the Homestead Act.
Land typically sold by the railroad companies was situated alongside the tracks and cheap to entice development along the route.

Because consumer culture had already been established and was prominent at the time, most of the goods a household used were purchased from a store.
Households did not produce anything beyond some food and crops to sell at markets and in the east.
In order to sell things in the west, where people were very spread out, mail-order catalogs were used to purchase and deliver goods from the east.
Railroads were used to transport goods long distances for little cost.


Settlers would not have been able to travel and procure land at such small, or free, prices if not for the federal land act or companies distributing unused land. 
Corporations were able to adapt and take advantage of a distant market and sell goods many miles away from the factories through the railroad network connecting the United States.