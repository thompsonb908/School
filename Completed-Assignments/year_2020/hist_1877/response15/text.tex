During the Great Depression, FDR said that \textquote{life was no longer free; liberty no longer real; men could no longer fulfil the pursuit of happiness.}
Another observer states: \textquote{Our democracy finds itself\ldots\ in a new age where not political freedom but social and industrial freedom is the most insistent cry.}
The terms social and industrial freedom are vague, there wasn't a concrete definition for them but FDR believed that we were redefining what freedom meant.
FDR's opinion of freedom was \textquote{greater security for the average man,} and saw the old ideas as only benefitting the economic elite.

To fix these issues and the Depression, FDR implemented the New Deal.
The two different conceptions of liberty were, freedom for private enterprise (or the individual) to act as they wanted without restraints, and socialized liberty in which the abundance of the American economy was shared.
People that opposed the New Deal, particularly many business owners, were sympathetic to the first view, and the democrats and reformers supported the latter view.

Republicans view the New Deal as an attack on Americanism, and saw it as an attempt to take away the rights of the people.
The believed that the New Deal was economically wasteful and was contributing to the duration of the Great Depression.
The Democrats believed that the wealthy were lengthening the gap between the wealthy and the poor, moving common people into a virtual serfdom for corporations.

The result of the 1936 election was the greatest lopsided victories ever in a presidential election. 
Roosevelt won 60\% of the popular vote and 523 of 531 electoral votes.
America clearly sided with FDR's view of freedom and liberty, leading to a reorientation of American politics moving Democrats into the dominant political force for the next 50 years.