The 1950's was a very affluent time period, it was the peak of the economic experience.
The economy grew at a steady rate, unemployment and inflation were very low during this time.
There were many new technological innovations that increased the productivity of households and the ease of accessing entertainment.
Electricity, central heating and plumbing became easily accessible and common in homes.

During this time period, the distribution of income was more favored for the bottom 90\%.
Previously, the share of income for the top 1\% was about 20\% of the total share of income, where the bottom 90\% only received 55\% of the total share of income.

There are many reasons that the poor were getting wealthier and the rich were not taking as much money, the main one being that much of the worlds industrial power was damaged during the war.
With Americas industries being one of the only that could produce at full capacity, the US was able to benefit from selling goods to other countries.

The government also funneled immense amounts of money into military and intelligence programs.
This influx of money funded more jobs and led to the development of new technologies like the internet and space programs, giving America a technological advantage over other countries.

Typically, a economic boom is followed by an economic bust and vice versa.
To prevent the economy from falling into a cycle of booms and busts and to ensure that everyone got their fair share of profits the government regulated the economy.