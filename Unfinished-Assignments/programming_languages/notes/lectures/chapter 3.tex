\section{Recursive Definitions}
New method to define languages using three steps:
\begin{enumerate}
    \item Specify the basic words (base case).
    \item Rules for constructing new words from ones already known (recursive case).
    \item Declare that no word except those constructed by following rules 1 and 2 are in the language.
\end{enumerate}

\subsection{Examples}
\subsubsection{Factorial Function}
\begin{enumerate}
    \item \(0! = 1\)
    \item \(n!=n(n-1)!\)
    \item No word except those constructed by following rules 1 and 2 are in the language.
\end{enumerate}
\subsubsection{Recusive Programs}
\begin{itemize}
    \item \(\Sigma+,\Sigma\{x\}\)
    \begin{enumerate}
        \item \(x\in\Sigma+\)
        \item If \(w\) is any word in \(\Sigma+\), then \(xw\) is in \(\Sigma+\).
        \item No word except those constructed by following rules 1 and 2 are in the language.
    \end{enumerate}
    \item \(\Sigma^*, \Sigma=\{x\}\)
    \begin{enumerate}
        \item \(\Lambda \in \Sigma^*\)
        \item If \(w\) is any word in \(\Lambda^*\), then \(xw\) is in \(\Sigma^*\).
        \item No word except those constructed by following rules 1 and 2 are in the language.
    \end{enumerate}
    \item S-ODD
    \begin{enumerate}
        \item \(x\in\text{S-ODD}\)
        \item if \(w\) is any word in S-ODD then \(xxw\) is in S-ODD.
        \item No word except those constructed by following rules 1 and 2 are in the language.
    \end{enumerate}
\end{itemize}